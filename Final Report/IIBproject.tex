% IIB project report
% Robert Hudson (rh689)

\documentclass{IIBproject}
% Use documentclass[wide]{IIBproject} to have narrower margins
\usepackage{setspace}



\bibliographystyle{ieeetr}
\pagestyle{empty}
% The next line sets 1.5 spacing
\onehalfspacing



\begin{document}


% Title Page
\author{Robert Hudson}
\supervisor{Dr Paul Robertson}
\title{Portable/Disposable System to Measure Liver Optical Backscatter}
\maketitle
\thispagestyle{empty}

\begin{titlepage}
\includegraphics[width=6.5cm]{Engineering.png}\par
    \begin{centering}
        \vspace{1cm}
        {\huge\bfseries Portable / Disposable System for Measuring Liver Optical Backscatter\par}
        \vspace{0.5cm}
        {\Large B-PAR10-1}\par
        \vspace{1cm}
        \includegraphics[height=5cm]{jesus.jpg}\par
        \vspace{1cm}
        {\Large\bfseries Robert Hudson}\par
        {\Large
        Jesus College\par
        \vspace{0.5cm}
        Supervisor: Dr Paul Robertson\par
        \vspace{0.5cm}
        Department of Engineering \par
        University of Cambridge \par
        \vspace{0.5cm}
        This report is submitted for the degree of \par
        {\em Master of Engineering}
        }

    \end{centering}
    \vfill

    {\footnotesize \noindent I hereby declare that, except where specifically indicated, the work submitted herein is my own original work.}\par
    \vspace{0.5cm}
    \begin{minipage}[t]{0.5\linewidth}
        {\small\textit{Signed:}}
      	\includegraphics[width=0.4\linewidth]{signature.pdf}
    \end{minipage}
    \hfill
    \begin{minipage}[t]{0.3\linewidth}
        {\small\textit{Date:} \today}
    \end{minipage}

\end{titlepage}

\pagenumbering{roman}
%% Summary
%\begin{abstract}
%Just a test Just a test Just a test Just a test Just a test Just a test 
%Just a test Just a test Just a test Just a test Just a test Just a test 
%No more than 100 words.
%\end{abstract}

%\newpage
%\tableofcontents
%\newpage
%\pagestyle{plain}
%\pagenumbering{arabic}
%
%
%\section{Background}

The liver is a vital organ within the body, responsible for metabolic processes, blood detoxification, and parts of the immune system \cite{liver_background}. The NHS performed 922 liver transplants in the year 2019-20 \cite{nhs_transplants}. This project concerns the design of a probe to assess the quality of donor organs and evaluate if they are suitable for transplantation. This probe is intended to replace an older product which gave good predictions for liver failure but had several issues. The new design needed to be:

\begin{itemize}[noitemsep, nolistsep]
\item Easy to use
\item Able to withstand being dropped
\item Inexpensive to replace in case of damage
\end{itemize}


\subsection{Liver Transplants}
One threat to the survival of liver transplant patients is hepatic steatosis. Macrovesicular steatosis occurs when a hepatocyte (liver cell) vacuole, which is made of fat, takes up much of the cell cytoplasm, squeezing the nucleus to the periphery (see figure \ref{fig: livers}). In the donor patient, this is a benign disease \cite{Imber2002}.\\

\begin{figure}[htb]
	\centering
	\begin{subfigure}[b]{0.4\linewidth}
		\includegraphics[width=\linewidth]{lean liver.PNG}
		\caption{Lean liver \cite{Bruno2008}}
		\label{}
	\end{subfigure}
	\begin{subfigure}[b]{0.4\linewidth}
		\includegraphics[width=\linewidth]{fatty liver.PNG}
		\caption{Fatty liver \cite{Bruno2008}}
		\label{fig: fatty liver}
	\end{subfigure}
	\caption{Comparison of a lean liver to a fatty liver. The white voids in B are fat deposits.}
	\label{fig: livers}
\end{figure}	


Issues arise when steatotic livers are transplanted. Whilst the exact mechanisms are not fully understood, the cold preservation process is linked with damage to the liver tissue, making it likely to fail in the recipient \cite{Imber2002}. Moderate to high levels of steatosis are associated with unacceptably high rates of primary non-function (PNF) and early allograft dysfunction (EAD). Both PNF and EAD are key causes of mortality in liver transplant patients, with patients having a one in three chance of dying if PNF occurs \cite{Robertson}.\\

It is currently unacceptable to transplant severely steatotic livers due to their high failure rate. Unfortunately, a method to quantify the level of steatosis quickly and accurately does not currently exist. The most common method to assess the quality of a donor liver is the opinion of the surgeon who removes the organ. However, this is subjective, and hence is inherently unreliable \cite{Robertson}. To gain a quantitative assessment, biopsies can be sent for histological analysis, but this is a slow process. The longer the liver is outside the body, the higher the risk of PNF and EAD, so this is not an ideal method. Previous studies \cite{McLaughlin2010} have attempted to assess the fat content of liver tissue using electrical and optical spectroscopy. The results from the optical spectroscopy led to the development of a probe \cite{Robertson} which gave a rapid assessment of the liver steatosis. 


\subsection{Original Probe}

\begin{figure}[htb]
	\centering
	\begin{subfigure}[b]{0.3\linewidth}
		\includegraphics[width=\linewidth]{original probe.PNG}
		\caption{Original probe \cite{Robertson}.}
		\label{fig: original probe}
	\end{subfigure}
	\begin{subfigure}[b]{0.5\linewidth}
		\includegraphics[width=\linewidth]{probe tip.PNG}
		\caption{Diagram of the tip of the original probe \cite{Robertson}.}
		\label{fig: probe tip}
	\end{subfigure}
	\caption{}
\end{figure}

This project focuses on the development of the original liver probe produced by Robertson et al. \cite{Robertson}. This probe used the optical backscatter properties of the liver tissue to assess its fat content. Figure \ref{fig: probe tip} shows a schematic of the tip of the original probe, which was pressed against the liver tissue. Two LEDs (one red, one infrared (IR)) were shone down an optical fibre to illuminate the liver tissue. Some of the light was scattered back towards the probe and was collected by another optical fibre, which led to a photodiode for detection. Lipid droplets within the tissue are a major source of scattering, so the more fat in the liver the greater the amount of backscattered light \cite{McLaughlin}. This meant that backscatter measurements could provide an indication of the liver fat content.\\

The probe progressed to clinical trials, where it was found that the correlation between its reading and fat concentration was poor (R\textsuperscript{2} = 0.456), but the correlation with PNF/EAD was strong \cite{Robertson}. The strong predictive power for PNF/EAD meant the probe could give a rapid, quantitative, near-patient test to indicate whether the liver will survive in the new patient. These promising results warranted the further development of the probe.\\

The probe had several issues, which this project aims to resolve. Due to the high-pressure environment of surgery, the surgeons often damaged the probe. Sources of damage included being dropped on the floor and being placed in preservation solution. This project will develop a new iteration of the original probe which is more robust to the stresses of the clinical environment.





%
%
%\section{System Requirements}

The system requirements were derived from an analysis of the original probe's weaknesses and from a meeting with Addenbrooke's surgeon Dr Chris Watson, who had experience of using the original probe.


\begin{enumerate}
\item \label{req: simple} \textbf{The system should be simple to use with little prior training}\\
The surgeon using the system will be under high levels of stress, so the system should be intuitive to use to ensure the surgeon does not accidentally use it wrongly. The system should provide a clear interface which does not require prior training to understand, as it cannot be assumed that every surgeon will have been taught how to use it.

\item \label{req: cheap} \textbf{The system should be cheap}\\
The system should be cheap compared to the overall cost of a liver transplant, which can be several thousand pounds. Any disposable parts of the system should be cheap so that it is not an issue to discard them if the they are damaged. Dr Watson suggested that the disposable parts of the system should cost no more than £45.

\item \label{req: robust} \textbf{The system should be mechanically robust}\\
The original probe frequently experienced accidental damage in the clinical environment, rendering it non-functional. Sources of damage included being dropped on the floor and being submerged in preservation fluid. Therefore, the new system should be designed to be robust to these sources of damage.

\item \label{req: small} \textbf{The probe should be small}\\
The probe should fit comfortably in the surgeon's hand.


\item \label{req: correlation} \textbf{The system's output should be highly correlated with the output of the original probe}\\
As this project concerns the design of the new probe before it can reach clinical trials, the results it produces should be highly correlated with controlled tests of the original probe, to ensure it will perform similarly in clinic. The comparison tests were carried out by performing a measurement at variable distances from a piece of A4 paper, as this simulates changing the backscatter of the liver.



\item \label{req: sterilise} \textbf{The system must be able to withstand the medical sterilisation process}\\
In order to be used in clinic, the probe must be able to be sterilised. The most common sterilisation is an autoclave, which use saturated steam under pressure to heat the device to 134\si{\celsius} for 3-3.5 minutes \cite{nhs_autoclave}. The system should be able to withstand the thermal and mechanical stresses associated with this. The system should also be made from materials able to withstand the chemical sterilisation process, which varies across NHS trusts.

\item \label{req: seal} \textbf{The system should be hermetically sealed}\\
A hermetic seal will ensure that the device is safe from comtamination from pathogens getting inside the device and not being sterilised. 

\item \label{req: biocompatible} \textbf{The system should be made from biocompatible materials}\\
Biocompatible materials are essential to ensure device safety and that it will pass the approval process.




\item \label{req: screen} \textbf{The system should have a screen}\\
The system should display the measurements on a screen so the surgeon knows the result of the measurement immediately. The screen should also display any menus to select other features and error messages so the surgeon can easily navigate the device interface.

\item \label{req: temperature} \textbf{The system should measure temperature}\\
The brightness of LEDs and responsivity of photodiodes varies with temperature, so a temperature measurement should be logged with each liver measurement to enable the system to be calibrated properly.

\item \label{req: memory} \textbf{The system should store measurements in non-volatile memory}\\
The system should have a source of non-volatile memory which can store at least 1000 measurements. A Real Time Clock (RTC) module will be required to time stamp the data, to make it clear which operation each measurement refers to.

\item \label{req: rs232} \textbf{The system should be able to output stored data to a PC}\\
The system should output data to a PC over an RS232 link, as the surgeons already have training in how to use RS232 logging software (for example, Termite \cite{termite}).


\end{enumerate}
%
%
%\section{System Architecture}

The system was subdivided into two devices: the ``base'' unit and the ``remote'' unit. The remote unit would perform the measurement on the liver and transmit the data to the base unit. Only the remote unit will need to be hermetically sealed and able to withstand the sterilisation process, as the base unit does not come in contact with biological tissues. Furthermore, all the expensive electronics such as the screen can be contained within the base, making the remote unit cheaper and hence more disposable. Also, only the remote unit will need to be mechanically robust to the clinical environment, so the base unit can be packaged less rigidly. 



\begin{figure}[htbp]
	\centering
	\includegraphics[width=\linewidth]{architecture.png}
	\caption{Block diagram of system architecture. Key sub-systems are highlighted in different colours: measurements (orange), power (green) and communications (yellow).}
	\label{fig: architecture}
\end{figure}

The remote unit will be packaged by casting the electronics in a resin, which will provide the hermetic seal and mechanical robustness. Because of this, it cannot be opened or have any sockets to connect cables. Therefore, it must communicate wirelessly with the base unit, and be battery powered with wireless charging. The wireless communications are required to operate over a range of several metres, as it is assumed that the base unit could be placed at the opposite side of the room to where the surgery is happening. Figure \ref{fig: architecture} shows the system block diagram with all the key sub-systems labelled.
%
%
%\section{Microcontroller}
A Peripheral Interface Controller (PIC) was chosen as the microcontroller for both the base and remote unit. PICs are versatile, with highly configurable pin assignments, which were suitable for sub-systems used in the product. Furthermore, they are easy to program in PICBASIC using the MicroCode Studio \cite{microcode_studio} and MPLAB IDE software packages. The free version of MicroCode Studio limits the range of devices which can be programmed, so the PIC16F688 (14-pin) \cite{pic16f688} was chosen for the remote unit and the PIC18F2550 (28-pin) \cite{pic18f2550} for the base unit. These had the right number of I/O pins for their respective unit, as well as some special features which will be discussed in later sections.\\

\begin{figure}[htb]
	\centering
	\includegraphics[width=0.6\linewidth]{pickit3.PNG}
	\caption{PIC connected to PICkit 3 programmer \cite{pickit3}.}
	\label{fig: pickit3}
\end{figure}

The PICkit 3 \cite{pickit3} was used to program the PICs. Figure \ref{fig: pickit3} shows the relevant connections required for programming. The PIC18F2550 had a USB interface, which took up the majority of \verb|PORTC|. When the PIC was tested on the breadboard, these USB pins could not operate as standard I/O pins as they pulled the pin high or low. The data sheet specified that USB functionality could be turned off by writing to the \verb|UCON| and \verb|UCFG| registers, but this did not work. This meant that the pins designated for USB were not used in this project.\\

%
%
%\section{Measurement System}
\subsection{Hardware}

\begin{figure}[htbp]
	\centering
	\includegraphics[width=\linewidth]{measurement.PNG}
	\caption{Circuit schematic of diffuse reflectance measurement system.}
	\label{fig: measurement schematic}
\end{figure}

Figure \ref{fig: measurement schematic} illustrates the circuit used to perform the backscatter measurements on the liver. Fundamentally, this circuit shines an LED at the liver, then produces an output voltage proportional to the amount of light reflected back. The clinical trials of the original probe \cite{Robertson} used two LEDs, one with a red wavelength (\SI{660}{\nano\metre}) and one with an infra-red wavelength (\SI{850}{\nano\metre}). Two photodiodes were also used, so over all 4 different measurements could be made (with each photodiode-LED combination). The results showed that the correlation with liver PNF did not depend on the wavelength or LED-photodiode spacing, so this was not a critical design requirement of the new probe. Therefore, an LED in the near infra-red (NIR) range was selected (Vishay TSHA4401 \cite{tsha4401} \SI{875}{\nano\metre}) along with a phototransistor also in the NIR range (Vishay BPW85B \cite{bpw85b} \SI{850}{\nano\metre}).\\

The left hand side circuit of figure \ref{fig: measurement schematic} provides a constant current to the LED. This is essential to ensure the measurements are repeatable, as the brightness of an LED is proportional to its current. The operation of the circuit will now be discussed. First consider the case where MOSFET Q1 has its gate driven low, so it is off. This means that negligible current will flow through the MOSFET, so the op-amp U1A has \SI{2.5}{\volt} at its non-inverting input. Assuming the op-amp is ideal (which means it has infinite gain and input impedance, and zero output impedance), the voltage at the inverting input will also be \SI{2.5}{\volt}. This means there is a constant \SI{2.5}{\volt} across R2, so it draws a constant \SI{11.4}{\milli\ampere}. Because the op-amp is assumed to be ideal, no current can go into the input pins, so the full \SI{11.4}{\milli\ampere} must be sourced from the op-amp output pin and hence pass through the LED D5. This creates a constant current source for the LED, assuming a constant \SI{2.5}{\volt} rail and that the ideal op-amp assumption is valid. When the MOSFET gate is driven high it turns on. This leads to the MOSFET having a very low impedance, so the non-inverting input is effectively tied to ground. Therefore R2 has no volts across it, and hence no current, so the LED turns off.\\

The MCP6002-I/SN \cite{mcp6002} was selected as the op-amp. This has a short circuit output current of \SI{23}{\milli\ampere}, so will comfortably be able to provide the desired \SI{11.4}{\milli\ampere}. Its input offset voltage is $\pm\SI{4.5}{\milli\volt}$, which leads to a potential current error of \SI{20}{\micro\ampere} (or 0.17\%), which is a negligible error. The input voltage noise density is \SI{28}{\nano\volt\per\sqrt{\hertz}} and the gain-bandwidth product is \SI{1}{\mega\hertz} so the worst case bandwidth is \SI{1}{\mega\hertz} (which assumes unity gain). This leads to an input noise voltage of \SI{28}{\micro\volt}, which again is negligible. The current source was tested for its invariance to temperature. When exposed to the heat fro ma hairdryer, the LED current only changed by 0.85\%, so there is confidence that the current source will be stable over the operating temperatures.\\

The light transmitted by the LED is then reflected off the target liver, and the received light is collected by phototransistor Q2. This produces a current proportional to the input light power. A transimpedance amplifier is used convert this to a readable voltage signal. Again assuming op-amp U1B is ideal implies that the voltage at the input pins are equal, and no current enters the pins. The phototransistor current $i_f$ therefore passes through the resistor R3, which leads to an output voltage signal $v_0 = 2.5 - i_fR_f$. The \SI{2.5}{\volt} rail at the non-inverting input acts to bias the op-amp, and was selected as half the voltage rails to ensure that the op-amp output signal could have maximum voltage swing before clipping at the supply rails.\\

\begin{table}[htbp]
	\centering
	\caption{Results for feedback resistor calibration experiments. The peak to peak signal is the difference between the output voltage when the LED was on and off, and the minimum signal is the output voltage when the LED was on.}
	\label{tab: tia feedback resistor}
	\begin{tabular}{|c|c|c|c|}
		\hline
		\textbf{LED Current} & \textbf{Feedback Resistor} & \textbf{Peak to Peak Signal} & \textbf{Minimum Signal}\\
		(mA)	&	(\si{\kilo\ohm})	&	(V)	&	(V)\\
		\hline
		\multirow{4}{*}{10}	&	2.2	&	0.93	&	1.20\\
						\cline{2-4}
						&	3.3	&	1.20	&	1.06\\
						\cline{2-4}
						&	4.7	&	1.60	&	0.62\\
						\cline{2-4}
						&	6.8	&	2.22	&	0.00\\
		\hline
		\multirow{4}{*}{20}	&	1.2	&	0.84	&	1.42\\
						\cline{2-4}
						&	1.5	&	1.07	&	1.15\\
						\cline{2-4}
						&	2.7	&	1.9	&	0.25\\
						\cline{2-4}
						&	3.3	&	2.09	&	0.00\\
		\hline
	\end{tabular}
\end{table}

The feedback resistor $R_f$ needed to be chosen to ensure good output characteristics. If it was too small, then the PIC's ADC would not be able to discriminate between changes in the light intensity.  The PIC16F688 uses a 10-bit ADC, so the maximum voltage resolution is \SI{4.88}{\milli\volt} when operated at \SI{5}{\volt}. The feedback resistor also cannot be too large, as this could lead to the op-amp saturating at the ground supply rail. Table \ref{tab: tia feedback resistor} shows the results of an experiment carried out to investigate the signal levels with different feedback resistors. The circuit was constructed on breadboard, with the LED and phototransistor soldered to stripboard to ensure they remained at a fixed distance of \SI{7.62}{\milli\metre}. This was then shone at a piece of A4 paper at a distance which led to the largest signal. The peak to peak signal is the difference between the transimpedance amplifier output voltage when the LED was on and off, and the minimum signal is the output voltage when the LED was on. Whenever the minimum signal was \SI{0}{\volt}, the amplifier had saturated so these values of $R_f$ are unsuitable. Therefore, the optimum $R_f = \SI{4.7}{\kilo\ohm}$, as this had the largest peak to peak signal without saturating. The system was tested using an LED current of \SI{20}{\milli\ampere} as well, but it was decided that this was too close to the op-amp output current limit to be a reliable, stable current source.\\

%LTSpice simulation

The transimpedance amplifier is a notoriously unstable circuit due to the phototransistor's junction capacitance. This can be stabilised by adding the capacitor C1 \cite{tia_stability}, which adds a zero into the feedback factor, compensating for the pole created by the phototransistor capacitance \cite{tia_stability}. The capacitor value was selected to be \SI{470}{\pico\farad}, as this gave a bandwidth of \SI{72}{\kilo\hertz} which is well above the required bandwidth of the circuit (which is \SI{10}{\kilo\hertz} at most), but is well below the gain-bandwidth product of \SI{1}{\mega\hertz} so it avoids the stability problems.\\

A button also needed to be designed for the probe, so the surgeon could indicate that they wanted to take a measurement. A standard push button could not be used because the remote unit had to be hermetically sealed. Therefore, a non-contact means of sensing a button press had to be devised. This could either be capacitive, sensing the finger like the touch screen on a mobile phone, or optical, by measuring light reflected back from the finger. It was found that the liver backscatter measurement system (figure \ref{fig: measurement schematic}) was discriminated well between the presence or absence of a finger at a range of a few centimetres, so this circuit was included twice within the remote unit to be used as a button. The downside to this method is that it is an active sensing system, requiring power to be supplied to the LED and phototransistor. This will limit the time the remote unit can operate over one full charge. \\

\begin{figure}[htbp]
	\centering
	\includegraphics[width=0.4\linewidth]{2-5v rail.PNG}
	\caption{Op-amp unity buffer used to generate \SI{2.5}{\volt} rail.}
	\label{fig: 2.5v rail}
\end{figure}

The measurement system used a \SI{2.5}{\volt} reference rail. This rail had to be stable as it provided the reference signal for the LED current source and transimpedance amplifier, so any variations in the \SI{2.5}{\volt} signal would lead to measurement errors. An op-amp unity buffer circuit (figure \ref{fig: 2.5v rail}) was used to generate this signal from a potential divider reference. This is more stable than directly using a potential divider, as any loading of the potential divider will decrease the reference voltage it provides. Because an op-amp draws negligible current, the resulting output voltage will be much more stable. The output will only remain stable the circuits it is connected to do not draw more than the op-amp's rated output current. The \SI{2.5}{\volt} rail is only used to supply current to a \SI{10}{\kilo\ohm} resistor (\SI{250}{\micro\ampere}) and op-amp input (negligible current), so the op-amp will not have to provide significant current. Additionally, the output will only be a fixed \SI{2.5}{\volt} reference if the \SI{5}{\volt} supply also remains stable, which is ensured by using a voltage regulator (see section \ref{power}).\\




\subsection{Firmware}
The system would give different readings depending on the background lighting conditions, as these would appear as an offset in the phototransistor current. To add resilience to background light, each measurement was the difference between the phototransistor reading when the LED was on (which gives information about the target liver and the background light) and the reading when the LED was off (which gives information about the background light). Assuming the phototransistor has a linear incident light intensity-current response, this difference will leave just the information about the liver with no dependence on background light. A series of readings was made (N=50) and the average taken, to further increase the resilience to noise. The PICBASIC code is given below.
\begin{lstlisting}
ir_led var PORTC.2
red_led var PORTC.0
phototransistor con 7       '' Analogue port used as input

reading var word
dark var word
light var word
N var byte
i var byte

mainloop:
reading = 0
dark = 0
light = 0

low red_led

for i = 1 to N
    low ir_led
    pause 1
    adcin phototransistor, light      '' read in light data
    high ir_led
    pause 1
    adcin phototransistor, dark      '' read in dark data
    
    '' Add current readings to sum
    '' Dark should be > light if measurement variation is due to our probe
    if dark >= light then
        reading = reading + (dark - light)
    else
        reading = 0
    endif
next i  

'' Compute mean
reading = reading / N

goto mainloop
\end{lstlisting}






\subsection{Testing}
\begin{figure}[htbp]
	\centering
	\includegraphics[width=\linewidth]{old probe comparison.png}
	\caption{Comparison of old and new probe readings when aimed at a sheet of A4 paper, at a separation of 1, 2, 3, 4 and 5 cm. The red line indicates the boundary between high- and low-fat livers measured by the old probe in clinical trials. Plotted on a log-log scale.}
	\label{fig: old probe comparison}
\end{figure}

The measurement system was first tested to measure the correlation between readings taken with the old probe and those taken with the new design. This was important, because the new design's readings need to be highly correlated with the old probe to ensure that the new probe will have the same predictive power in clinical trials. The probes give an output in arbitrary units which is simply the ADC output of the PIC. These are arbitrary units, and will be different for the two designs due to different component selection and circuit design. The two probes had their correlation tested by measuring the output at fixed distances from a sheet of A4 paper. This will give different intensities of backscattered light, so can be used for calibration. Figure \ref{fig: old probe comparison} shows the outcome of this experiment. The R\textsuperscript{2} correlation coefficient is 0.9859, so the probes are highly correlated as desired. \\

\begin{figure}[htbp]
	\centering
	\includegraphics[width=\linewidth]{liquids.png}
	\caption{Measurements made by the probe on different liquids, with a liquid depth of \SI{2}{\centi\metre} and probe range of \SI{3}{\centi\metre}.}
	\label{fig: liquids}
\end{figure}

The measurement system was then tested to determine its response to lipid concentration. Readings were taken from shining the probe at water, semi-skimmed milk (2\% fat), and full-fat milk (4\% fat). The milk will have suspended lipid droplets which can be used as a proof-of-concept to model probe's response to fat concentration. It will also include other biological molecules such as proteins. Figure \ref{fig: liquids} shows that the probe reading does increase for the liquids with higher fat concentration.







%
%
\section{Wireless Charging and Power}\label{power}

\subsection{Hardware}

The remote unit was designed to run from a \SI{5}{\volt} supply. \SI{3.3}{\volt} was not suitable as the LEDs had a forward voltage of \SI{1.5}{\volt}, so a \SI{3.3}{\volt} supply would potentially be too small for the current source depicted in figure \ref{fig: measurement schematic}. A set of batteries had to be selected which would provide this voltage. The batteries had to be small to ensure the overall remote unit design was small enough to fit in the surgeon's hand, and needed to be rechargeable because of the hermetic seal. Two different chemistries were available from RS Components \cite{rs} in a small, rechargeable package: Lithium ion (Li-ion) and Nickel Metal Hydride (NiMH). The NiMH chemistry was more suitable due to the increased safety. Li-ion batteries are highly sensitive to over/under-voltage and high temperatures, and can blow up if these limits are exceeded \cite{batteries}. This requires complex protection circuitry, and the battery may not survive being placed in an autoclave. NiMH batteries are much more resilient to over- and under-charging, so are much safer than Li-ion. A \SI{2.4}{\volt} 80 mAh RS-Pro NiMH button cell \cite{rs_pro_batteries} was suitable. Three of these were used to give a nominal \SI{7.2}{\volt} supply, which gave a margin of 44\% for when the cells started to discharge.\\

%The remote unit was designed to run from a \SI{5}{\volt} supply. \SI{3.3}{\volt} was not suitable as the LEDs had a forward voltage of \SI{1.5}{\volt}, so a \SI{3.3}{\volt} supply would potentially be too small for the current source designed in figure \ref{fig: measurement schematic}. A set of batteries had to be selected which would provide this voltage. The batteries had to be small to ensure the overall remote unit design was small enough to fit in the surgeon's hand, and needed to be rechargeable because of the hermetic seal. Two different chemistries were available from RS Components \cite{rs} in a small, rechargeable package: lithium ion (Li-ion) and Nickel Metal Hydride (NiMH). The NiMH chemistry was more suitable due to the increased safety. Li-ion batteries are highly sensitive to over/under-voltage and high temperatures, and can blow up if these limits are exceeded \cite{batteries}. This requires complex protection circuitry and may mean that the battery will not survive being placed in an autoclave. NiMH are much more resilient to over and under charging, so are much safer than Li-ion. They also have a higher energy density \cite{batteries}. Nevertheless, they are lower voltage so more cells will be required to provide the desired \SI{5}{\volt}, have a long charging time, and they have a high self-discharge rate \cite{batteries}. This is not such an issue for this application, as the remote unit can remain charging until it is required for an operation, and then it will only be used for a short time before being placed on charge again. A \SI{2.4}{\volt} \SI{80}{\milli\ampere\hour} RS-Pro NiMH button cell \cite{rs_pro_batteries} was suitable. Three of these were used to give a nominal \SI{7.2}{\volt} supply, which gave a margin of 44\% for when the cells started to discharge.\\
%Cut?

A power converter was then selected to drop the \SI{7.2}{\volt} battery voltage to the required \SI{5}{\volt} rail. A Low-Dropout (LDO) voltage regulator was deemed suitable for this application, as it would provide a stable voltage rail. The advantage over a buck converter is the smaller device size and design simplicity, but this comes at the cost of worse efficiency. Furthermore, the output voltage will not have the high-frequency ripple associated with switch-mode converters, which was desirable to create a stable analogue rail to ensure that there were no measurement errors. The MCP1702T-5002E/CB \cite{mcp1702} in a SOT-23A package was suitable. This was a \SI{5}{\volt} regulator with a rated output current of \SI{250}{\milli\ampere}. This was well above the \SI{50}{\milli\ampere} maximum current required by the remote unit (see section \ref{power budget}). The quiescent current was \SI{2}{\micro\ampere}, which was negligible compared with the remainder of the circuit, so the regulator will not affect the battery lifetime. The regulator will dissipate a maximum of $(\SI{7.2}{\volt}-\SI{5}{\volt})\times\SI{50}{\milli\ampere}=\SI{110}{\milli\watt}$. The SOT-23A package has a junction-to-air thermal resistance of \SI{336}{\celsius\per\watt}, so the maximum temperature increase is \SI{37}{\celsius}.\\

\begin{figure}[htb]
	\centering
	\includegraphics[width=\linewidth]{h bridge charger.PNG}
	\caption{Circuit schematic for the H-Bridge inductive wireless charger. The left-hand side circuit is included in the base unit, and the right-hand side circuit is included in the remote unit.}
	\label{fig: charger schematic}
\end{figure}
	

Inductive charging was an appropriate method to recharge the batteries. This involved placing two air-cored coils in close proximity, applying an AC voltage to the primary, and rectifying the voltage across the secondary. This was a similar design to that used in other hermetically sealed low-power products, such as an electric toothbrush charger \cite{wireless_power_review}. Figure \ref{fig: charger schematic} shows the circuit schematic, which uses two PMOS and NMOS pairs in a H-bridge arrangement to generate the AC voltage. The two half bridges were driven with inverted signals, so the voltage across the load alternated between \SI{\pm 5}{\volt}. The series capacitor was used to create a resonant circuit with the inductor, which increased the voltage across the primary coil and smoothed the input square wave to a sinusoid across the inductor. The capacitor was designed to give the circuit a resonant frequency equal to the drive frequency to get the maximum gain possible. The series resistor was included to limit the current through the circuit. An ideal series LC circuit has zero impedance at resonance, so it looks like a short circuit to the external circuit. The PMOS transistors had a current rating of \SI{230}{\milli\ampere}, so a series resistance of \SI{22}{\ohm} limited the current to \SI{227}{\milli\ampere}.\\

The primary coil generated an alternating magnetic field due to the sinusoidal voltage across it. This coupled with the secondary coil to induce an alternating voltage. The coupling of air-cored transformers is very poor, so a large voltage was required across the primary coil to ensure that the secondary coil received enough volts. The resonant circuit aided this. The alternating voltage across the secondary was then rectified to a DC voltage by a diode bridge. The NSR05F20NXT5G Schottky barrier diode \cite{original_diode} was used in the diode bridge. These were chosen because of their very low forward voltage of \SI{0.2}{\volt}, which made sure that the required secondary voltage was kept as small as possible. Their reverse leakage was \SI{2}{\micro\ampere}, which was also good. They were rated to \SI{20}{\volt} and \SI{500}{\milli\ampere}, which was well below the expected operating conditions of \SI{10}{\volt} and \SI{10}{\milli\ampere}. The battery voltage was \SI{7.2}{\volt}, so the total required voltage across the secondary, including the two diode drops, was \SI{7.6}{\volt}.\\

\begin{figure}[htb]
	\centering
	\includegraphics[width=0.8\linewidth]{coil inductance.png}
	\caption{Coil inductance for different numbers of coil turns.}
	\label{fig: coil inductance}
\end{figure}
\newpage
There are many empirical laws for the inductance of coils. Wheeler's formula \cite{Wheeler} for circular planar coils is:
\begin{equation}\label{eq: solenoid}
L (\si{\micro\henry})= \frac{r^2N^2}{8r + 11d}
\end{equation}
where $N$ is the number of turns, $r$ is the average coil radius in inches, and $d$ is the coil depth in inches. The circular coils constructed for the wireless charger had an inner diameter of \SI{5}{\centi\metre} and an average depth of \SI{0.5}{\centi\metre} (this varied depending on the number of turns). Substituting these geometry parameters into equation \ref{eq: solenoid} implies that $L=\num{9.65e-8}N^2$. The coil inductance was then experimentally determined by measuring the resonant frequency of an LC circuit with known capacitance. The frequency of the input signal was varied until the maximum output voltage across a parallel LC tank was measured. The inductance followed a quadratic dependence on the number of turns (see figure \ref{fig: coil inductance}), as is to be expected from equation \ref{eq: solenoid}. The coefficient is \num{6e-8}, which gives reasonable agreement to the prediction from equation \ref{eq: solenoid}. Equation \ref{eq: solenoid} performs better at small numbers of turns, which could be because it assumes a coil with a single layer of turns. As more turns are added, the coil geometry transitions from being planar to being toroidal. Furthermore, the construction of the coils meant that there was not a uniform distribution of the coil turns, which could also influence the measured inductance.\\

\begin{figure}[htb]
	\centering
	\includegraphics[width=0.3\linewidth]{coil experiment.png}
	\caption{Setup for the coil experiments.}
	\label{fig: coil experiment setup}
\end{figure}
\begin{figure}[htb]
	\centering
	\includegraphics[width=\linewidth]{v2 circular coupling.PNG}
	\caption{Measured coupling coefficient of circular coils as a function of coil separation.}
	\label{fig: circular coupling}
\end{figure}

The performance of the proposed charging circuit was investigated. First, the coupling of the coils was determined as a function of the coil separation. The experiment was performed using two circular planar coils which were centred on the same out-of-plane axis. Figure \ref{fig: coil experiment setup} illustrates the experiment setup. The primary coil was driven by the H-bridge circuit in figure \ref{fig: charger schematic}, and the secondary coil had its open circuit voltage measured. A capacitor was used to make the system resonant at \SI{100}{\kilo\hertz}, using the equation:
\begin{equation}
C = \frac{1}{(2\pi f_\text{res})^2L}
\end{equation}
If the transformer were ideal, the secondary coil voltage would be $V_2 = V_1 \times N_2 / N_1$, where $V_1$ is the voltage across the primary coil, and $N_2 / N_1$ is the secondary to primary turns ratio. To account for the poor coupling in an air-cored transformer, the coupling coefficient $k$ is introduced, so $V_2 = k \times V_1 \times N_2 / N_1$. The reduction in coupling is because the air-core has a low permeability, so the flux from the primary is not strongly linked to the secondary, leading to a large leakage flux. The experiment was repeated for different turns ratios, all giving similar values of the coupling coefficient. The experimentally determined coupling coefficients are plotted in figure \ref{fig: circular coupling}, which shows a coupling coefficient of 0.28 at a separation of \SI{10}{\milli\metre}.\\


\begin{figure}[htb]
	\centering
	\includegraphics[width=0.6\linewidth]{charger simulation.PNG}
	\caption{LTspice simulation of the charger circuit.}
	\label{fig: charger simulation}
\end{figure}

The circuit was simulated in LTspice \cite{ltspice}, with the simulation schematic shown in figure \ref{fig: charger simulation}. A voltage source with series resistance was used to model the H-bridge output. The circuit was tested on breadboard, so the surface mount NSR05F20NXT5G diodes could not be used. Instead, a through-hole diode was selected for testing purposes (Vishay UF4001-E3/54 \cite{tht_diode}), which was modelled in the LTspice simulation, so the results could be compared with the experimental results. The coupling between the coils was set to 0.28 to model \SI{10}{\milli\metre} coil separation. The inductors and batteries had an equivalent series resistance modelled additionally. The simulation suggested that the primary coil should have an RMS voltage of \SI{44.4}{\volt}, and the secondary coil should have an RMS voltage of \SI{8.24}{\volt}. This led to an average battery charging current of \SI{13.1}{\milli\ampere}, which was suitable to recharge the remote unit. \\

\begin{table}[htb]
	\begin{center}
	\caption{Current measurements (in milliamps) across a 7.2V battery load for various combinations of primary and secondary coil (with the coils touching).}
	\label{tab: coil combinations}
	\begin{tabular}{|l|*{4}{c|}}
		\hline
		\backslashbox{\textbf{Primary Turns}}{\textbf{Secondary Turns}} & \textbf{50} & \textbf{100} & \textbf{160} & \textbf{200} \\
		\hline
		\textbf{50} & - & 3.97 & 5.32 & 3.32 \\
		\hline
		\textbf{100} & 4.46 & - & 5.22 & 4.75 \\
		\hline
		\textbf{160} & 3.95 & 4.15 & - & 3\\
		\hline
	\end{tabular}
	\end{center}
\end{table}

The charger was tested to see the effect of changing the number of turns in the primary and secondary coils (see table \ref{tab: coil combinations}). The best combinations of primary and secondary coils were the 50/160 and 100/160 combinations. The maximum charging current was \SI{5.3}{\milli\ampere} when the coils were touching, which was much less than the simulated value at \SI{10}{\milli\metre} separation. At \SI{10}{\milli\metre} separation, the charger could only deliver around \SI{0.5}{\milli\ampere}. This could be because of the presence of parasitic elements within the circuit damping the resonance and reducing the magnitude of the currents. \\ 



The initial tests indicated that the charger would not deliver sufficient current to charge the batteries quickly, as the recommended current to charge the batteries in 16 hours was \SI{8}{\milli\ampere} \cite{rs_pro_batteries}. Therefore, the supply voltage was stepped up to \SI{12}{\volt}, as this would lead to a larger voltage at the secondary side, so a greater current could flow. The circuit in figure \ref{fig: charger schematic} needed to be adapted to the \SI{12}{\volt} supply. A gate driver needed to be designed for the PMOS to provide it with a \SI{12}{\volt} gate signal rather than the \SI{5}{\volt} signal the PIC generates. Because the PMOS gate voltage is referred to the \SI{12}{\volt} rail, if a \SI{5}{\volt} signal were applied to the gate then the PMOS would remain on, as the gate-source voltage (\SI{7}{\volt}) would be greater than the threshold voltage. This would result in a large shoot-through current when the corresponding NMOS was on, as it would look like a short to ground. This would damage the MOSFETs and the supply, as well as not allowing the charger to operate properly. \\

\begin{figure}[htb]
	\centering
	\includegraphics[width=0.7\linewidth]{mcp1407.PNG}
	\caption{Schematic of MCP1406/07 MOSFET driver internal circuitry \cite{mcp1407}.}
	\label{fig: mcp1407}
\end{figure}

To avoid this problem, two MOSFET driver ICs (Microchip's MCP1406 and MCP1407 \cite{mcp1407}) were selected to replace the H-bridge. These ICs were designed to drive the gate of a power MOSFET, but are also suitable for this application because they supply a large current (up to \SI{6}{\ampere}) and a large voltage (up to \SI{18}{\volt}). Figure \ref{fig: mcp1407} demonstrates that the internal circuitry of the MCP1406/07 takes an input CMOS/TTL signal and uses this to control a push-pull CMOS inverter. Therefore, the H-bridge MOSFETs can be replaced by these ICs, as they take care of the gate drive. The drivers come in two varieties: the inverting MCP1406 and the non-inverting MCP1407. Choosing one of each (for the different sides of the H-bridge) means only one input signal is required from the PIC, freeing up an I/O pin for other uses. The propagation times for both devices are the same (\SI{40}{\nano\second}), so there is no risk of dangerous shoot-through currents, assuming they are placed close enough together on the PCB so that the delay time in the PCB traces is negligible.\\

\begin{figure}[htb]
	\centering
	\includegraphics[width=0.8\linewidth]{mosfet drivers charger.PNG}
	\caption{Primary side of the charging circuit using the MCP1406/07 MOSFET drivers.}
	\label{fig: mosfet drivers charger}
\end{figure}

The schematic for the primary side is shown in figure \ref{fig: mosfet drivers charger}. Three capacitors were included so the resonant circuit could be finely tuned. A \SI{22}{\ohm} resistor was used to limit the current to \SI{500}{\milli\ampere}, as this was the maximum current the power supply could provide. The circuit provided \SI{2.5}{\milli\ampere} of charging current to the three \SI{2.4}{\volt} batteries when tested on breadboard with the coils separated by \SI{10}{\milli\metre}. This was reflective of the real-life conditions, and the \SI{2.5}{\milli\ampere} charging current would fully charge the 80 mAh batteries in \SI{32}{\hour}. The batteries' recommended charging current is \SI{8}{\milli\ampere}, so while the current is a little low, this will not be a large problem as it is expected that the remote unit will be left on constant charge.\\
\newpage
Using the \SI{12}{\volt} supply voltage resulted in \SI{112}{\volt} across the primary coil and \SI{51}{\volt} across the secondary (under open circuit conditions with \SI{10}{\milli\metre} separation). This was greater than the NSR05F20NXT5G voltage rating of \SI{20}{\volt}, so new diodes were selected for the bridge rectifier to handle the higher voltage. The NXP PMEG10020ELRX Schottky barrier diode \cite{new_diode} was rated to \SI{100}{\volt} and \SI{2}{\ampere}, so was suitable for the application. It had a low forward voltage of \SI{0.4}{\volt} and a reverse leakage current of \SI{10}{\nano\ampere} at \SI{25}{\celsius}, which will lead to good performance.\\




\begin{figure}[htb]
	\centering
	\includegraphics[width=0.6\linewidth]{battery current sensor.PNG}
	\caption{Battery current sensor.}
	\label{fig: battery current sensor}
\end{figure}

A current sensor was included to measure the remote unit charging and discharging currents. A \SI{1}{\ohm} resistor (R2 in figure \ref{fig: charger schematic}) was placed in series with the batteries to develop a voltage proportional to the current through the batteries. R2 was placed between BT2 and BT3 so it sat at \SI{2.4}{\volt}, as it was important to ensure that the voltage input to the amplifier was not close to the amplifier supply rails, as this would result in clipping. The voltage across R2 was used as the input to a differential amplifier (figure \ref{fig: battery current sensor}). The output voltage $v_o = 2.5 - 47 i_\text{bat}$, where $i_{bat}$ is the current through R2. The gain (47) was given by the ratio of resistors R20 and R17 in figure \ref{fig: battery current sensor}. The \SI{2.5}{\volt} rail was used as a pseudo-ground to allow the output to swing in both the positive and negative direction without clipping. The output signal was input to a PIC ADC pin to allow its value to be determined by the software.\\



A circuit was also designed to monitor the voltage of the batteries, so as to detect an under-voltage condition. As the batteries discharged, their voltage also reduced. If the battery voltage fell below \SI{5.15}{\volt} (the voltage regulator rated voltage plus the dropout voltage), then the regulator would be unable to provide \SI{5}{\volt} to the remaining circuit. If the \SI{5}{\volt} rail was reduced, then the LED current and phototransistor output signal would also be reduced, which would make the measurements inaccurate. Therefore, the PIC needed a way to detect an under-voltage condition, so that it could stop taking measurements and turn off any high-power systems, to prevent further battery discharge.\\

\begin{figure}[htb]
	\centering
	\includegraphics[width=0.8\linewidth]{battery voltage sensor.PNG}
	\caption{Battery under-voltage sensor.}
	\label{fig: battery voltage sensor}
\end{figure}

Figure \ref{fig: battery voltage sensor} shows the circuit used to detect the under-voltage condition. U6B is an op-amp which acts as a comparator with hysteresis, which is required to prevent oscillations. When the batteries provide current, their output voltage is reduced due to their internal impedance. This means that when the comparator indicates an under-voltage condition, the PIC will reduce the current load, so the battery voltage will rise. This may be above the comparator threshold, so the PIC will turn the current back on, leading to an oscillatory loop. With the hysteresis, the batteries must be charged above the upper threshold $v_{TH}$ before the comparator will change state. This threshold can be designed to give a stable voltage above \SI{5}{\volt} when the batteries are loaded. The upper threshold voltage was designed to be \SI{7.2}{\volt} and the lower threshold $v_{TL}$ was set to \SI{4.96}{\volt}. The potential divider R21 and R22 halves the battery voltage, because otherwise it would be too large for a \SI{5}{\volt} comparator. The resistors R23, R24 and R26 were chosen to set the threshold voltages according to the design equations \cite{hysteresis}:%(Rewrite) Explain loop more
\begin{multicols}{2}
\begin{equation}
\frac{v_{TL}}{v_{TH} - v_{TL}} = \frac{R26}{R23}
\end{equation}

\begin{equation}
\frac{v_{TL}}{5 - v_{TH}} = \frac{R24}{R23}
\end{equation}
\end{multicols}

Diode D7 was included because the PIC pin the circuit was connected to, RA3, was also the $\overline{\text{MCLR}}$ pin used by the PICkit3 programmer. This pin was raised to a high voltage during PIC programming. Diode D7 clamps the op-amp output to \SI{5}{\volt} to prevent damage during programming. The \SI{33}{\kilo\ohm} resistor allows the programmer to adjust the pin voltage independent of the op-amp, so the op-amp does not tie it high or low. The $\overline{\text{MCLR}}$ CONFIG bit had to be cleared in the PIC program to allow this pin to be used as a digital input. \\

Once the PIC detects an under-voltage condition, it will turn off all the outputs which consume power. The wireless transmitter module was designed with a PMOS switch on its Vcc rail, so that it could be fully switched off by the PIC. The PIC then transitions to its low-frequency oscillator mode, which uses a \SI{31}{\kilo\hertz} clock rather than the standard \SI{8}{\mega\hertz} clock. This reduces the quiescent current from \SI{2}{\milli\ampere} to \SI{30}{\micro\ampere} \cite{pic16f688}. Periodically, the PIC will check if the battery voltage has been restored, and if it has, then it will turn the device back on.
 



\subsection{Firmware}
The wireless charger requires a PWM signal with frequency $f=\SI{100}{\kilo\hertz}$ and duty cycle $D=0.5$. The PIC18F2550 has a built-in Capture/Compare/PWM (CCP) module \cite{pic18f2550}. This allows the PWM signal to be generated in hardware, rather than having to worry about precisely timing the software. A software implementation would have been difficult, as the PIC must do many other tasks as well as the PWM signal generation (such as listening for any incoming messages on the receiver). This would result in a variable delay time before the PWM could next be toggled. The CCP pin could be set up as PWM at the beginning of the program and would continue to generate the signal regardless of what the rest of the program was doing.\\

PICBASIC contains a built-in command to access the CCP pins, \verb|hpwm|. However, because the free version of MicroCode Studio does not support the long data type, the highest frequency \verb|hpwm| could deliver is \SI{32.767}{\kilo\hertz} \cite{picbasic_pro}. The wireless charger was designed to run at \SI{100}{\kilo\hertz}, so the PWM signal needed to be faster than \verb|hpwm| would allow. Therefore, a series of register writes were programmed to set up the CCP pin. The hardware was designed to use CCP module 2 on pin 24, which was multiplexed to \verb|PORTB.3|. CCP2 could be multiplexed to either \verb|PORTC.1| or \verb|PORTB.3|, so the appropriate CONFIG bit was set with the command \verb|CONFIG CCP2MX = OFF|.\\

The PIC18F2550 datasheet \cite{pic18f2550} detailed the required register writes to set up CCP2 as a PWM pin. First, the PWM period was written to the \verb|PR2| register (Timer2 period register). The PWM period is given by the equation:
\begin{equation}
\text{PWM Period} = [\text{PR2}+1] \times 4 \times T_\text{osc} \times (\text{TMR2 Prescale})
\end{equation}
where $T_\text{osc}$ is the oscillator period, so for an \SI{8}{\mega\hertz} clock $T_\text{osc}=\SI{125}{\nano\second}$. The factor of 4 is present because each PIC operation takes four clock cycles to be executed, so the effective clock speed is $f_\text{osc}/4$. For a TMR2 prescaler setting of x1, the required \verb|PR2| value was \verb|19| for a \SI{100}{\kilo\hertz} PWM signal. The duty cycle was then defined by the equation:
\begin{equation}
\text{PWM Duty Cycle} = (\text{CCPR2L:CCP2CON\textless5:4\textgreater}) \times T_\text{osc} \times (\text{TMR2 Prescale Value}) \label{eq: duty cycle}
\end{equation}
Where \verb|CCPR2L| is the CCP2 register low byte and \verb|CCP2CON<5:4>| are bits 5 and 4 of the CCP2 control register. Equation \ref{eq: duty cycle} implies $(\text{CCPR2L:CCP2CON\textless5:4\textgreater}) = \num{4e6}$, so \verb|0x0A| was written to \verb|CCPR2L| and \verb|0b00| was written to \verb|CCP2CON<5:4>|. After this was done, \verb|TRISB.3| was cleared to set the pin to a digital output. The TMR2 prescaler was set to x1 and Timer2 was enabled by writing \verb|0x04| to \verb|T2CON| (Timer2 control register). Finally, \verb|CCP2CON<3:2>| was set to \verb|0b11| in order to set CCP2 to PWM mode. This generated a stable \SI{100}{\kilo\hertz} square wave.

\begin{lstlisting}
#config
    CONFIG CCP2MX = OFF
#endconfig

'' Set up PWM to run in hardware at 100kHz, D = 0.5
PR2 = 19
CCPR2L = %00001010
CCP2CON.5 = 0
CCP2CON.4 = 0
TRISB.3 = 0
T2CON = %00000100
CCP2CON = %00001111
\end{lstlisting}




\subsection{Testing}
\begin{figure}[htb]
	\centering
	\includegraphics[width=\linewidth]{v2 rectangular coupling.png}
	\caption{Coupling coefficient for rectangular and circular coils.}
	\label{fig: rectangular coupling}
\end{figure}


The charger was tested at \SI{100}{\kilo\hertz} using a 100-turn rectangular coil for the primary and a 150-turn rectangular coil for the secondary. The coils measured \SI{70 x 35}{\milli\metre}. Rectangular coils were required to maximise the coil area for the probe casing geometry. Figure \ref{fig: rectangular coupling} demonstrates that the coupling between the rectangular coils was identical to the circular coils used in the breadboard tests, so the results for the circular coils will have the same order of magnitude if repeated with rectangular coils.\\

\begin{figure}[htb]
	\centering
	\includegraphics[width=\linewidth]{secondary current.PNG}
	\caption{Battery charging current waveform at \SI{100}{\kilo\hertz} and a separation of \SI{10}{\milli\metre}.}
	\label{fig: secondary current}
\end{figure}

\begin{figure}[h!]
	\centering
	\includegraphics[width=\linewidth]{v2 charging current.png}
	\caption{Charging current delivered to \SI{7.2}{\volt} battery load at different coil separations.}
	\label{fig: current distance}
\end{figure}
\newpage
The charging system was first tested to measure the current delivered to the \SI{7.2}{\volt} battery load at different coil separations. A \SI{3.2}{\pico\farad} capacitor was used to ensure the primary coil (with \SI{840}{\micro\henry} inductance) operated near resonance at \SI{100}{\kilo\hertz}. Figure \ref{fig: secondary current} shows that the current waveform observed in the batteries was a rectified sinusoid, as expected. Figure \ref{fig: current distance} demonstrates that the current-distance characteristics show a similar trend to the coupling-distance characteristics. The recommended charging current for the batteries is \SI{8}{\milli\ampere} \cite{rs_pro_batteries}, which is exceeded at distances less than \SI{28}{\milli\metre}. The minimum coil separation is \SI{10}{\milli\metre}, which includes the plastic casing surrounding the coils.\\

\begin{figure}[h!]
	\centering
	\includegraphics[width=\linewidth]{v2 resonant frequency.png}
	\caption{Charging current delivered to the \SI{7.2}{\volt} battery load where the resonant frequency is varied. The coils had a \SI{10}{\milli\metre} separation.}
	\label{fig: current resonance}
\end{figure}

The effect of changing the resonant frequency of the system was then measured. The \SI{3.2}{\pico\farad} capacitor was replaced with a series of other capacitors, and the PIC was reprogrammed to drive the coil at the LC resonant frequency. Figure \ref{fig: current resonance} shows that the maximum charging current is achieved at \SI{55}{\kilo\hertz} using a \SI{10}{\nano\farad} capacitor, so the system will operate at this frequency.\\ 

%The charging current decreases at low and high frequencies. %******WHY THE DECREASE AT LOW/HIGH FREQUENCIES************
%Therefore, the system will be designed to operate at \SI{55}{\kilo\hertz}, as this will provide the maximum current.\\

\begin{figure}[h!]
	\centering
	\includegraphics[width=\linewidth]{v2 frequency.png}
	\caption{Charging current delivered to the \SI{7.2}{\volt} battery load as a function of drive frequency for a fixed \SI{55}{\kilo\hertz} LC resonant frequency. The coils had a \SI{10}{\milli\metre} separation.}
	\label{fig: current frequency}
\end{figure}

The current delivered to the batteries at the \SI{55}{\kilo\hertz} resonance was \SI{36.8}{\milli\ampere}, which would fully charge the 80 mAh batteries in 2 hours 10 minutes. However, this is 4.6x the recommended charging current, so it could damage the batteries if they are charged at this current for a long time. By moving away from the resonant frequency, the voltage across the primary coil will decrease, and hence the current delivered to the batteries will also decrease. Therefore, by changing the drive frequency the charging current can be controlled, which is shown in figure \ref{fig: current frequency}. This means that the remote unit can be kept on a constant ``trickle charge'' when it is not in use, and then when a surgery is being prepared, the charger can change to a ``fast charge'' regime to ensure the device is fully charged for when it needs to be used. The recommended trickle charge for the batteries is \SIrange{2.4}{4}{\milli\ampere} \cite{rs_pro_batteries}.\\

\begin{table}[htb]
	\begin{center}
	\caption{Wireless charger power measurements.}
	\label{tab: efficiency}
	\begin{tabular}{|c|c|c|c|c|}
	\hline
	\textbf{Drive} & \textbf{Charging} & \textbf{Input} & \textbf{Output} & \\
	\textbf{Frequency} & \textbf{Current} & \textbf{Power} & \textbf{Power} & \textbf{Efficiency} \\
	(\si{\kilo\hertz}) & (\si{\milli\ampere}) & (\si{\watt}) & (\si{\watt}) & (\%)\\
	\hline
	55.7 & 39.4 & 2.76 & 0.306 & 11.1\\
	\hline
	45.5 & 8.69 & 1.05 & 0.067 & 6.42\\
	\hline	
	\end{tabular}
	\end{center}
\end{table}

The efficiency was then measured. The input voltage was \SI{12}{\volt}dc, and the RMS input current was measured across the \SI{22}{\ohm} series resistor in the RLC circuit. The output voltage was measured across the batteries when they were open circuit, and the RMS output current was measured across the \SI{1}{\ohm} current sense resistor. Table \ref{tab: efficiency} shows the measured efficiency when the circuit operates in resonance, and when it delivers the recommended \SI{8}{\milli\ampere} current. The efficiency is small (around 10\%), which is to be expected as the system uses air-cored coils which will have lots of leakage, and the coils have some resistive loss. Modern Qi wireless phone chargers have an efficiency of 70\%, although these use a more complex control system \cite{wireless_power_review}. This system has been designed to be simple and low-cost, so the efficiency penalty is justified. The base unit is mains powered, and its overall power consumption is small, so the low efficiency is not an issue. Nevertheless, the efficiency is better when the circuit is driven in resonance than when it is off resonance.\\ %*********WHY???********** 

When the batteries were loaded, their output voltage dropped from \SI{7.2}{\volt} to \SI{5.2}{\volt}. This means that there is only a small operating range where the LDO can provide a stable \SI{5}{\volt} rail. Upon reflection, it may have been better to use a buck-boost converter, as this can maintain an output voltage which is higher than the input, so the lifetime of the device will increase.
% Maybe take this out??






\subsection{Power Budgets}\label{power budget}
\subsubsection{Remote Unit}

\begin{table}[htb]
	\begin{center}
	\caption{Current drawn by major components on the remote unit. These are expressed as a difference between the current measured when the PIC was idle and the current when the corresponding output pin was high.}
	\label{tab: remote current}
	\begin{tabular}{|c|c|c|c|c|}
	\hline
	& & & Tx module & Tx module\\
	\textbf{Component} & Red LED & IR LED & (low output) & (high output)\\
	\hline
	\textbf{Current (mA)} & 14.3 & 23.9 & 0 & 0.6 \\
	\hline
	\end{tabular}
	\end{center}
\end{table}

The PIC drew \SI{9.4}{\milli\ampere} when it was idle and running on an \SI{8}{\mega\hertz} clock. This gives the device a ``stand-by'' lifetime of \SI{8.5}{\hour}, as 80 mAh batteries are used. When the clock speed was reduced to \SI{31}{\kilo\hertz}, the current reduced to \SI{7.5}{\milli\ampere}. The \SI{1.9}{\milli\ampere} reduction helps to ensure that the batteries do not become fully discharged. Table \ref{tab: remote current} shows the currents drawn by the other components in the remote unit. In practice, the transmitter module drew negligible current, so there was no need to include the PMOS switch on its power rails. The current drawn when the IR LED was on was \SI{23.9}{\milli\ampere}. This was greater than the \SI{11.4}{\milli\ampere} design current from the circuit in figure \ref{fig: measurement schematic}. When the voltage across the \SI{220}{\ohm} resistor was measured, the correct current flowed through it, so the reason for the additional \SI{12.4}{\milli\ampere} is unknown at this time. Further tests using laboratory equipment may yield an explanation for this.\\

The total energy stored in the batteries is $E_\text{Bat} = \SI{7.2}{\volt}\times\SI{80}{\milli\ampere\hour}\times 80\% = \SI{1659}{\joule}$. The factor of 80\% is given by the batteries' datasheet \cite{rs_pro_batteries} and accounts for the fact that the batteries will only hold a stable \SI{7.2}{\volt} output for 80\% of their total capacity. The number of readings per full battery charge was then calculated. This assumed that the button LED was flashed 10 times in a button read, the measurement LED was flashed 50 times to probe the liver, the LEDs were on for \SI{1}{\milli\second} each flash, a \SI{1200}{\bit\per\second} baud rate was used, and the data was transmitted three times to add redundancy. It was also assumed that the button would only be polled once every \SI{100}{\milli\second}.\\

\begin{table}[htb]
	\begin{center}
	\caption{Energy used by different subsystems in the remote unit.}
	\label{tab: remote energy}
	\begin{tabular}{*{5}{|c}|}
	\hline
	\textbf{System} & Button LED & Measurement LED & Wireless Tx & Idle time\\
	\hline
	\textbf{Energy (mJ)} & 1.07 & 5.34 & 23.1 & 4.70	 \\
	\hline
	\end{tabular}
	\end{center}
\end{table}

The calculated energies (table \ref{tab: remote energy}) were then summed to determine the number of readings which could be taken per full battery charge. When the probe was not in use, the button would activate every \SI{100}{\milli\second} and then the system would enter a delay period. This resulted in a stand-by time of 6.86 hours. If the probe was in constant use, so every button poll resulted in a measurement being taken, then 39,100 measurements could be taken, which corresponds to a lifetime of 5.95 hours. As the probe will be left on trickle charge when it is not in use, these lifetimes will be suitable for the probe to retain its charge during the transplant surgery.



\subsubsection{Base Unit}

\begin{table}[htb]
	\centering
	\caption{Current drawn by the major components on the base unit.}
	\label{tab: base current}
	\begin{adjustwidth}{-.5in}{-.5in}
	\begin{center}
	\begin{tabular}{|c|c|c|c|}
	\hline
	\multicolumn{2}{|c}{\textbf{5 V devices}} & \multicolumn{2}{|c|}{\textbf{3.3 V devices}}\\
	\hline
	\textbf{Component} & \textbf{Worst-case current (mA)} & \textbf{Component} & \textbf{Worst-case current (mA)} \\
	\hline
	PIC & 6 & LCD backlight & 100 \\
	\cline{3-4}
	LCD logic & 4 & \textbf{Total} & \textbf{100}\\
	\cline{3-4}
	LCD & 0.5 & \multicolumn{2}{c}{} \\
	\cline{3-4}
	Wireless Rx & 6 &  \multicolumn{2}{c|}{\textbf{12 V devices}} \\
	\cline{3-4}
	Wireless Tx & 12.5 & \textbf{Component} & \textbf{Worst-case current (mA)}  \\
	\cline{3-4}
	I\textsuperscript{2}C pullups & 4.5 & Wireless charger & 375 \\
	EEPROM & 3 & \SI{5}{\volt} converter & 37.6 \\
	RTC & 0.4 & \SI{3.3}{\volt} converter & 100 \\
	\cline{3-4}
	Push Buttons & 1 & \textbf{Total} & \textbf{512.6}  \\
	\hline
	\textbf{Total} & \textbf{37.9}  \\
	\cline{1-2}
	\end{tabular}
	\end{center}
	\end{adjustwidth}
\end{table}

The base unit required two voltage converters to generate a \SI{5}{\volt} and \SI{3.3}{\volt} rail. LDO regulators were selected due to their simplicity and their existing use in the remote unit. Although they will be inefficient, because the base is mains powered this will not be a problem. Each regulator needed to be able to provide the worst-case current to the components connected to those supply rails, as detailed in table \ref{tab: base current}. The large currents and voltage drops in the regulators led to a large power dissipation, which may lead to overheating in some packages.\\

\begin{figure}[htb]
	\centering
	\begin{subfigure}[b]{0.35\linewidth}
		\includegraphics[width=\linewidth]{parallel ldos.png}
		\caption{Parallel topology}
		\label{fig: parallel ldos}
	\end{subfigure}
	\begin{subfigure}[b]{0.5\linewidth}
		\includegraphics[width=\linewidth]{series ldos.png}
		\caption{Series topology}
		\label{fig: series ldos}
	\end{subfigure}
	\caption{Possible base voltage regulator topologies.}
	\label{fig: base ldos}
\end{figure}

The \SI{3.3}{\volt} rail could be created by stepping down either the \SI{12}{\volt} or the \SI{5}{\volt} rail (see figure \ref{fig: base ldos}). Both topologies led to the same total conversion loss of \SI{1.14}{\watt}, but the parallel topology was chosen because the \SI{5}{\volt} converter had a smaller output current, so it was less likely to reach the maximum available output current. To prevent overheating, the converters were set a maximum temperature rise of \SI{100}{\celsius} when operating at full output current. This meant that the junction-to-air thermal resistance had to be less than \SI{268}{\celsius\per\watt} for the \SI{5}{\volt} converter and \SI{86.2}{\celsius\per\watt} for the \SI{3.3}{\volt} converter. \\

\begin{table}[htb]
	\centering
	\caption{Key properties of the LDO voltage regulators used in the base unit.}
	\label{tab: ldo properties}
	\begin{adjustwidth}{-.5in}{-.5in}
	\begin{center}
	\begin{tabular}{|c|c|c|c|c|c|}
%	\hline
%	\textbf{Property} & \textbf{MCP1702-5002} & \textbf{MCP1755-3302}\\
%	\hline
%	\textbf{Package} & TO-92 & SOT-223-5\\
%	\hline
%	\textbf{Output Voltage (\si{\volt})} & 5 & 3.3\\
%	\hline
%	\textbf{Thermal Resistance (\si{\celsius\per\watt})} & 131.9 & 62\\
%	\hline
%	\textbf{Operating Temperature (\si{\celsius})} & 60 & 79 \\
%	\hline
%	\textbf{Max. Output Current (\si{\milli\ampere})} & 250 & 300\\
%	\hline
	\hline
	& & \textbf{Output} & \textbf{Thermal} & \textbf{Operating} & \textbf{Max. Output} \\
	& & \textbf{Voltage} & \textbf{Resistance} & \textbf{Temperature} & \textbf{Current}\\
	\textbf{Device} & \textbf{Package} &  (\si{\volt}) &  (\si{\celsius\per\watt}) &  (\si{\celsius}) &  (\si{\milli\ampere})\\
	\hline
	\textbf{MCP1702-5002} & TO-92 & 5 & 131.9 & 60 & 250 \\
	\hline
	\textbf{MCP1755-3302} & SOT-223-5 & 3.3 & 62 & 79 & 300 \\
	\hline
	\end{tabular}
	\end{center}
	\end{adjustwidth}
\end{table}

The MCP1702-5002 \cite{mcp1702} in a TO-92 package was a suitable \SI{5}{\volt} regulator, and the MCP1755-3302 in a SOT-223-5 package was a suitable \SI{3.3}{\volt} regulator. Their properties are summarised in table \ref{tab: ldo properties}. They will operate at \SI{80}{\celsius} at most, but this is below the maximum operating temperature of \SI{150}{\celsius}, so they will not be damaged. They can both provide the required currents.\\

%The MCP1702-5002 \cite{mcp1702} in a TO-92 package had a thermal resistance of \SI{131.9}{\celsius\per\watt} and a maximum output current of \SI{250}{\milli\ampere} so was a suitable \SI{5}{\volt} converter. Additionally, it is the same converter used in the remote unit. This component would operate at \SI{60}{\celsius}, so would feel hot but since the maximum operating temperature was \SI{150}{\celsius} then the device would not be damaged. The MCP1755-3302 \cite{mcp1755} in a SOT-223-5 package had a thermal resistance of \SI{62}{\celsius\per\watt}, so would operate at \SI{79}{\celsius} under the worst case load. Again, this was below the maximum operating temperature of \SI{150}{\celsius}. The maximum output current was \SI{300}{\milli\ampere}, which again was above the expected current.\\
%Thermals can be reduced a bit

It should be noted that the currents given in this section are the worst-case, as the converters must be designed to provide up to the worst-case current before failing. The expected currents are much smaller, so the total power consumption will be less than that predicted above. This is estimated to be around \SI{2}{\watt}, \SI{0.66}{\watt} of which is wasted in the voltage regulators.


%
%
%\section{Communications}

The wireless communications system needed to be reliable over the length of the operating theatre where the system will be used. This was estimated to be \SI{4}{\metre}. The Quasar QAM-TX3 \cite{qam-tx} and QAM-RX10-433 \cite{qam-rx} transmitter/receiver modules were used to implement the wireless link. These had all the signal processing required for a radio link integrated into a single module. Fully integrated modules were selected because they were low-cost and simplified the design greatly, as RF effects did not need to be considered. The Quasar modules allowed the raw data to be input to the data pin in baseband, then modulated this signal onto a \SI{433}{\mega\hertz} carrier using on-off keying (OOK). This was received by the receiver which demodulated the signal and output the baseband signal at its data pin. The modules could work up to \SI{3}{\kilo\bit\per\second}, which was a suitable speed for the small packets of data which were sent by the remote unit. They operate in the \SI{433}{\mega\hertz} band, which is an unlicensed ISM band in the UK \cite{ism_band}. A \SI{100}{\pico\farad} capacitor was used between the power rails of both modules to smooth out any ripple in the power supply. The transmitter used a short length of wire for an antenna. Due to the high output power (1 dBm \cite{qam-tx}), this led to a strong enough signal to be picked up by the receiver several metres away. \\
%Take out decoupling cap bit

%The modules operated using amplitude modulation (AM), rather than frequency modulation (FM) or phase modulation. AM is more susceptible to additive noise because the data is encoded into the amplitude of the carrier, whereas the amplitude of an FM signal does not relate to the data. Nevertheless, because the transmit power is high and the devices will operate over a short distance, the signal will dominate any noise or interference sources. This means that an AM transmitter is suitable due to its reduced complexity and hence smaller cost.\\

The simplest way to send data between the PIC and the transceivers was to use the \verb|serin2| and \verb|serout2| command. These use the RS-232 communications protocol. RS-232 is an asynchronous serial communication standard, originally designed to communicate between computers and modems \cite{rs232}. An RS-232 connection was also used for the serial link with the PC. To communicate with the PC, an RS-232 to USB module was used to convert the PIC signals to a signal which could be read by a PC from a USB port.\\

Each RS-232 data packet starts with a `0' start bit, which is followed by 8 data bits, a parity bit, and finally, an optional two stop bits \cite{rs232}. For this system, no parity bit was sent and one stop bit was used, following the `8N1' format. The \verb|serout2| command uses the syntax: \verb|SEROUT2 DataPin, Mode, [Data List]|. The \verb|DataPin| argument sets which PIC pin the data will be transmitted from. The \verb|Mode| argument controls how the data is sent. Bits 0-12 set the baud rate, which in this case was set to 1200 baud (\verb|Mode<12:0> = 813| \cite{picbasic_pro}). Bit 13 was set to `0' to indicate no parity bit. Bit 14 selected whether the PIC outputs signals in true form or inverted form, and was set to `1' for the inverted mode. Bit 15 was set to `0' to ensure that the data pin was always driven. Finally, the data which should be sent was specified in square brackets.\\

Preliminary tests found that several measures needed to be employed to improve the reliability of the wireless transmission. When no data was being transmitted, the dynamic threshold of the receiver reset to the background noise level, which meant that there was a short delay time before the threshold was re-established at the right level to demodulate the signal properly. Therefore, a series of `U' characters were transmitted. The ASCII value of the character `U' is \verb|`0b01010101'|, so this toggles between the high and low states to set the threshold properly. The \SI{433}{\mega\hertz} band is used for many devices (for example car key-fobs \cite{qam-rx}), so the receiver may pick up spurious transmissions which do not contain the liver probe data. The receiver was always listening, so to ensure that noise or interference were not picked up and decoded, a string was prefixed to the data. The receiver would wait until it received this prefix before starting to save data, as only then would it be sure that the signal was coming from the remote unit. For initial tests, the string \verb|`liver'| was prefixed to the data. \\

The relevant data were the liver reflectance measurement and the temperature. These were sent as four-digit decimal numbers in ASCII. To add resilience to decoding errors, a checksum was included with the data. Three checksum functions were proposed:
\begin{enumerate}
\item \label{check: binary xor} \textbf{Binary XOR} - Split the data word into two bytes, and perform an exclusive or operation between the two bytes to give a one-byte checksum.
\item \label{check: decimal xor} \textbf{Decimal XOR} - Split the four-digit decimal representation into two two-digit numbers (i.e. $X_4X_3X_2X_1$ becomes $X_4X_3$ and $X_2X_1$). Perform an exclusive or operation between the binary representation of these two-digit decimal numbers.
\item \label{check: decimal sum} \textbf{Decimal Sum} - Sum the four decimal digits.
\end{enumerate}

\begin{table}[htb]
	\begin{center}
	\caption{Mean and variance of the three checksum functions, assuming a uniform distribution for the data value between $[0,1023]$.}
	\label{tab: checksums}
	\begin{tabular}{|c|c|c|c|}
		\hline
		\textbf{Function} & \textbf{Mean} & \textbf{Variance} & \textbf{Average number of repeated sums}\\
		\hline
		\ref{check: binary xor} - Binary XOR & 127.5 & 5461 & 4.0\\
		\ref{check: decimal xor} - Decimal XOR & 48.8 & 857 & 9.5\\
		\ref{check: decimal sum} - Decimal Sum & 13.3 & 26 & 36.6\\
		\hline
	\end{tabular}
	\end{center}
\end{table}

The three functions had their mean and variance computed using a Python script (see table \ref{tab: checksums}). In this application, checksums are used to detect errors but not for error correction. Therefore, an ideal checksum would assign a different output for each input argument, as this will make it more likely for an error to be detected. The binary XOR has the best performance, as the binary operation naturally uses all 8 bits of the output byte, so it has the largest variance. The decimal operations are biased towards the smaller byte values, with the decimal sum being the worst as it can only access a range of $[0,36]$. The binary XOR was selected as the checksum. The data was also repeated to ensure that if the receiver accidentally missed a packet, then it was not lost. The full transmitter command is given below:

%with results presented in table \ref{tab: checksums}. In this application, checksums are used to detect errors but not for error correction. Therefore, the checksum should ideally assign a different output for each input argument, as this means that there will be fewer matching inputs for any checksum, so the error is more likely to be detected. The binary XOR has the best performance. It has a mean equal to the mean of all the possible checksum outputs so is centred, and has a large variance. It also only assigns each checksum value to an average of 4 possible input values, making it more likely that an error can be detected. This good performance is because the binary operation naturally uses all 8 bits of the output byte. The decimal operations are biased towards the smaller byte values, with the decimal sum being the worst as it can only access a range of $[0,36]$. The binary XOR was selected as the checksum. The data was also repeated to ensure that if the receiver accidentally missed a packet, then it was not lost. The full transmitter command is given below:\\

\begin{lstlisting}
''  Constant to define serout mode 8N1 at 1200 baud
baud_1200 con 813 | %0100000000000000
wireless_tx var PORTC.4				''  Set up an alias for transmitter pin
reading var word 					''  Liver reflectance measurement
temperature var word					''  Temperature measurement
checksum var byte

''  Bitwise XOR between upper and lower byte of reading
checksum = (reading / 256) ^ (reading // 256)

serout2 wireless_tx, baud_1200, [rep ``U''\10, ``liver '', dec4 reading, checksum, dec4 temperature, dec4 reading, checksum, dec4 temperature]
\end{lstlisting}

The \verb|dec4| modifier outputs the value stored in the variable as a four-digit decimal number, and the \verb|rep| modifier sends a repeated character \verb|\n| times. The base unit used the \verb|serin2| command in a similar fashion, with the main difference being the \verb|wait(``x'')| modifier, which will wait until it receives the string ``\verb|x|'' before saving data to the following variables. The code is given below:

\begin{lstlisting}
wireless_rx var PORTA.6                            ''  Set up an alias for receiver pin

serin2 wireless_rx, baud_1200, [wait(``liver''), dec4 reading, checksum, dec4 temperature]

''  Compare the received checksum to the checksum calculated from the reading. Error handling should go here.
if ((reading / 256) ^ (reading // 256)) != checksum then
    reading = 0
endif
\end{lstlisting}






\begin{figure}[htb]
	\centering
	\includegraphics[width=\linewidth]{dropout.png}
	\caption{Receive module output when a square wave signal is applied to transmit module.}
	\label{fig: dropout}
\end{figure}

The modules were initially tested with a square wave to ensure the receiver output the correct signal (see figure \ref{fig: dropout}). Initially, there is a noise region which indicates that the transmit module is not sending any data. The dynamic threshold will be low in this region, so any background noise will cause a transition across the threshold. After this, the square wave is observed. However, after roughly \SIrange{90}{110}{\milli\second}, the receiver output falls to zero for approximately \SIrange{60}{80}{\milli\second}. Only after this dropout region is reliable communication re-established. This dropout region occurred regardless of the baud rate and data. No reason could be found for this on the data sheet or through experimentation, but since the timing of the dropout could be predicted accurately, its effects could be mitigated. The remote unit was programmed to send \SI{200}{\milli\second} of character ``U'' before sending any real data, to ensure that the no data was lost in the dropout region.\\

\begin{figure}[htb]
	\centering
	\includegraphics[width=\linewidth]{v2 range.png}
	\caption{Proportion of 500 sent data packets which were received correctly.}
	\label{fig: range}
\end{figure}

The system was then tested to determine its maximum reliable range. This experiment was carried out indoors, to emulate the operating theatre environment. Both the transmitter and receiver were at ground level. 500 data packets were sent from the remote unit, and the number which were received by the base unit were recorded. Figure \ref{fig: range} demonstrates that the maximum reliable range of the system is \SI{14}{\metre}, and above this many packets can be lost. The sharp decrease at \SI{15}{\metre} could be due to fading effects caused by multiple reflections and scattering in the indoor environment, as the communication is re-established at \SI{16}{\metre}. Nevertheless, since the system only needs to communicate over an estimated \SI{4}{\metre}, it will perform well in practice.\\









%
%
%\section{Other Features}

\subsection{Temperature Sensor}
\begin{figure}[htbp]
	\centering
	\includegraphics[width=0.4\linewidth]{temperature sensor.PNG}
	\caption{Temperature sensor circuit schematic.}
	\label{fig: temperature sensor}
\end{figure}

A thermistor based temperature sensor (figure \ref{fig: temperature sensor}) was included with the remote unit. This was to ensure that measurements could be calibrated for any temperature dependent effects which may be observed, and to measure the temperature within the autoclave. Because the remote unit will be packaged in a resin, it is unknown what the internal temperature will be when it is exposed to the heat of the autoclave. The temperature sensor allows this to be measured and to work out if it will cause any issues for the electronics. The Vishay NTCLG100E2103JB \cite{thermistor} thermistor was selected, which had a nominal resistance of \SI{10}{\kilo\ohm} at \SI{25}{\celsius}. This was placed in a Wheatstone bridge arrangement with three fixed \SI{10}{\kilo\ohm} resistors, which would give a voltage imbalance $\Delta v$ when the thermistor was not at \SI{25}{\celsius}. This voltage was input to a differential amplifier, to yield an output signal which could be read by the PIC's ADC. The differential voltage $\Delta v$ is given by:\\
\begin{equation}
\Delta v = \frac{\delta R}{4 R}V
\end{equation}
where $\delta R$ is the change in resistance of the thermistor, and $R=\SI{10}{\kilo\ohm}$. The output voltage $v_o$ is then given by applying the differential amplifier gain:\\
\begin{equation}
v_o = 2 \frac{R_1}{R} \Delta v =R_1 \frac{\delta R}{2 R^2}V
\end{equation}
where $R_1$ is the value of feedback resistors R11 and R12, and $V=\SI{5}{\volt}$ is the supply voltage. Note that a \SI{2.5}{\volt} rail is used as a pseudo-ground to allow the output to swing positive and negative.\\

The sensor was designed to have a range of \SIrange{5}{150}{\celsius}, to cover the minimum operating temperature of just below room temperature \SI{20}{\celsius} up to the maximum temperature of the autoclave, \SI{134}{\celsius} \cite{nhs_autoclave}. Using \SI{6.8}{\kilo\ohm} feedback resistors gave an output voltage of \SI{5}{\volt} at \SI{5}{\celsius} and \SI{0.831}{\volt} at \SI{150}{\celsius}. At the high range, the thermistor only gives a very small change in $\delta R$ per degree of temperature change, so the sensor will not have a high resolution at these temperatures. For example, the output voltage at \SI{145}{\celsius} is \SI{0.835}{\volt}. The PIC's ADC has a minimum voltage resolution of \SI{4.88}{\milli\volt}, which means that the temperature resolution will be just larger than \SI{5}{\celsius} at high temperatures. As the high range is only used to understand the thermal properties of the autoclave and not to calibrate the measurements, this resolution will be acceptable. In comparison, at room temperature the resolution is much greater (for example a \SI{0.423}{\volt} change as temperature decreases from \SIrange{25}{20}{\celsius}, giving a temperature resolution of \SI{0.058}{\celsius}).\\





\subsection{LCD}
\begin{figure}[htbp]
	\centering
	\includegraphics[width=0.5\linewidth]{lcd.PNG}
	\caption{LCD interface circuitry.}
	\label{fig: lcd schematic}
\end{figure}

PICBASIC had a built-in command \verb|lcdout| which could be used to control an LCD with a Hitachi 44780 controller \cite{picbasic_pro}. To this end, an LCD module with an equivalent controller was selected. A 4x20 character matrix was desired to enable the screen to display all the information it needed to. The Displaytech 204-A-CC-BC-3LP \cite{lcd} was selected. This used the Samsung KS0076B driver, which had an equivalent instruction set to the Hitachi 44780, meaning it could be controlled using the \verb|lcdout| command. Figure \ref{fig: lcd schematic} shows the LCD interface circuitry. The R/$\overline{\text{W}}$ pin was tied to ground, as the LCD only needed to be written to. $V_o$ set the operating voltage for the LCD, which in turn controlled the contrast of the display. The LCD datasheet recommended biasing this at \SI{0.2}{\volt}, but experimental results showed that the best contrast was achieved when $V_o$ was biased close to ground. Therefore, the optimum experimental bias voltage was implemented by a potential divider using the two closest standard resistor values, to give a bias of \SI{0.045}{\volt}. The remaining pins were data signals which were connected to PIC pins. The LCD could operate using either a 4- or 8-bit parallel interface \cite{picbasic_pro}. In this application, the 4-bit interface was selected to minimise the number of PIC pins which would be used by the LCD. The \verb|lcdout| command required that these 4 data signals were connected to either the upper or lower nybble of a single I/O port. Only PORTA\textless0:3\textgreater was suitable, as all the remaining ports had pins which were required for other functions, or unimplemented bits in the PORT.\\

The \verb|lcdout| command allowed simple control of the LCD. First, several \verb|define|s were set to configure the compiler. Then, commands could be written to the LCD by sending the byte \verb|$FE| before the corresponding command in the PICBASIC manual \cite{picbasic_pro} or LCD datasheet \cite{lcd}. Characters could be written by looking up the character code in the LCD's character map (specified in the datasheet \cite{lcd}) or by simply providing PICBASIC with the required string. A delay was required between each write operation to ensure the LCD was ready to receive another instruction, otherwise an error would occur. A simple program is included below, and the output is shown in figure \ref{fig: lcd}.

\begin{lstlisting}
' Set LCD bus size (4 or 8 bits)
DEFINE LCD_BITS 4
' Set number of lines on LCD
DEFINE LCD_LINES 4
' Set command delay time in us
DEFINE LCD_COMMANDUS 1500
' Set data delay time in us
DEFINE LCD_DATAUS 44
' Set LCD Data port
DEFINE LCD_DREG PORTA
' Set starting Data bit (0 or 4) if 4-bit bus
DEFINE LCD_DBIT 0
' Set LCD Register Select port
DEFINE LCD_RSREG PORTC
' Set LCD Register Select bit
DEFINE LCD_RSBIT 5
' Set LCD Enable port
DEFINE LCD_EREG PORTA
' Set LCD Enable bit
DEFINE LCD_EBIT 4

LCD_DELAY con 1

'Need delay for LCD to boot
pause 500            
' Clear display move cursor to 2nd line 6th character
LCDOUT $FE, 1, $FE, $C0+6, "Welcome"         
pause LCD_DELAY
 ' Jump to third line 4th character
LCDOUT $FE, $94+4, "Liver Probe"                
\end{lstlisting}

\begin{figure}[htbp]
	\centering
	\includegraphics[width=0.4\linewidth]{welcome.jpg}
	\caption{LCD welcome screen}
	\label{fig: lcd}
\end{figure}





\subsection{Memory}
\begin{figure}[htbp]
	\centering
	\includegraphics[width=0.3\linewidth]{memory.PNG}
	\caption{EEPROM interface circuitry.}
	\label{fig: memory schematic}
\end{figure}

The base unit required some non-volatile memory to store measurements for the surgeon before they could download them to a computer. The EEPROM was designed to store a minimum of 1000 readings. Each reading had several components: the reading ID (\SI{1}{\byte}), month (\SI{1}{\byte}), day (\SI{1}{\byte}), hour (\SI{1}{\byte}), minute (\SI{1}{\byte}), measurement value (\SI{2}{\byte}), temperature reading (\SI{2}{\byte}). This led to a total reading size of \SI{9}{\byte}. This meant the memory required to store 1000 readings was $\SI{9}{\kilo\byte}=\SI{72}{\kilo\bit}$. The Microchip 24FC256-I/ST \cite{memory} was a \SI{256}{\kilo\bit} EEPROM chip which was selected. This allowed 3555 readings to be stored. The circuit in figure \ref{fig: memory schematic} was designed according to the data sheet recommendations \cite{memory}. The device communicated over an I\textsuperscript{2}C interface so it was connected to the PIC's SDA and SCL lines. It could have one of two addresses, which were set by the address pin A2. A \SI{10}{\kilo\ohm} pull-down resistor (R10) was used to pull this pin low. The resistor R9 is not placed, and is used to place an extra footprint on the PCB which would allow A2 to be pulled high if the designer desired. \SI{2.2}{\kilo\ohm} pull-up resistors were used on the I\textsuperscript{2}C lines, as this allowed the I\textsuperscript{2}C to run up to \SI{400}{\kilo\hertz} \cite{memory}.\\

The EEPROM chip was accessed using the PICBASIC \verb|i2cread| and \verb|i2cwrite| commands. The EEPROM allowed for sequential reads, so multiple bytes of data could be read out from one I\textsuperscript{2}C instruction, with the data saved in an array. To ensure the correct EEPROM address was used, a counter was stored in the PIC's internal EEPROM to record how many readings are stored in the external EEPROM. This was required because when the base unit powers on, it needs to know how many readings are already in EEPROM otherwise they will be overwritten.\\





\subsection{Real Time Clock}
\begin{figure}[htbp]
	\centering
	\includegraphics[width=0.4\linewidth]{rtc.PNG}
	\caption{RTC interface circuitry.}
	\label{fig: rtc schematic}
\end{figure}

The Microchip MCP7940N-I/SN \cite{rtc} was selected as the RTC module for the base unit. The circuit in figure \ref{fig: rtc schematic} was laid out according to the data sheet \cite{rtc} recommendations. A \SI{3}{\volt} CR2032 battery was used as a backup power supply for the module, to ensure the clock would continue timing even if the base power were removed. The module communicates over an I\textsuperscript{2}C interface, so it was connected to the PIC's SDA and SCL lines. The MFP output is not used.





\subsection{Buttons}
\begin{figure}[htbp]
	\centering
	\includegraphics[width=0.6\linewidth]{switches.jpg}
	\caption{Breakout board for push buttons.}
	\label{fig: buttons}
\end{figure}

The base unit required some buttons to interact with the menu. Four buttons were required, termed ``MENU'', ``OK'', ``UP'', and ``DOWN''. The 2-1825910-7 tactile switch \cite{buttons} available from Farnell was suitable, and was used with \SI{15}{\kilo\ohm} pull-up resistors. Due to a lack of available pins on the PIC, the ``OK'' and ``DOWN'' buttons were connected to the ``ICSPDAT'' and ``ICSPCLK'' lines, respectively. These are used for in-circuit serial programming, and the inclusion of the switch circuits does not cause an issue because the pull-up resistor allows the programmer to take control of the line whenever it requires. The ``MENU'' button was initially connected to an external interrupt pin ``INT2'', but after testing was changed to the ``$\overline{\text{MCLR}}$'' pin. When in normal operation, the base unit spends most of its time listening for data coming from the remote unit. When the ``MENU'' button is pressed, however, the base unit should immediately transition to the menu screen. PICBASIC does not have a convenient method to interrupt the \verb|serin2| command, so the ``MENU'' button was changed to the ``$\overline{\text{MCLR}}$'' pin to hard reset the unit when the button was pressed. The buttons could be read by a simple digital read to the corresponding PIC pin. A breakout board (figure \ref{fig: buttons}) was designed for the buttons to be soldered to, which in turn could be mounted to the base unit's casing by M2 bolts.\\










%
%%\section{Hardware Design}

A Peripheral Interface Controller (PIC) was chosen as the microcontroller for both the base and remote unit. PICs are versatile, with highly configurable pin assignments, which was suitable for the different types of sub-system which would be used in the product. Furthermore, they are easy to program in PICBASIC using the MicroCode Studio and MPLAB IDE software packages. PICBASIC provides a high level way to access the PIC's instruction set, with commands available which combine many complex assemble code instructions. The free version of MicroCode Studio limits the range of devices which can be programmed, so the PIC16F688 \cite{pic16f688} was chosen for the remote unit and the PIC18F2550 \cite{pic18f2550} for the base unit. These had the right number of I/O pins for their respective unit, as well as some special features which will be discussed in later sections. PICs are also relatively cheap, which is a good property for the disposable remote unit.\\

\subsection{Measurement System}
\begin{figure}[h]
	\centering
	\includegraphics[width=\linewidth]{measurement.PNG}
	\caption{Circuit schematic of diffuse reflectance measurement system}
	\label{fig: measurement schematic}
\end{figure}

Figure \ref{fig: measurement schematic} illustrates the circuit used to perform the backscatter measurements on the liver. Fundamentally, this circuit shines an LED at the liver, then produces an output voltage proportional to the amount of light reflected back. The clinical trials of the original probe \cite{Robertson} used two LEDs, one with a red wavelength (\SI{660}{\nano\metre}) and one with an infra-red wavelength (\SI{850}{\nano\metre}). Two photodiodes were also used, so over all 4 different measurements could be made (with each photodiode-LED combination). The results showed that the correlation with liver PNF did not depend on the wavelength or LED-photodiode spacing, so this was not a critical design requirement of the new probe. Therefore, an LED in the near infra-red (NIR) range was selected (Vishay TSHA4401 \cite{tsha4401} \SI{875}{\nano\metre}) along with a phototransistor also in the NIR range (Vishay BPW85B \cite{bpw85b} \SI{850}{\nano\metre}).\\

The left hand side circuit of figure \ref{fig: measurement system} provides a constant current to the LED. This is essential to ensure the measurements are repeatable, as the brightness of an LED is proportional to its current. The operation of the circuit will now be discussed. First consider the case where MOSFET Q1 has its gate driven low, so it is off. This means that negligible current will flow through the MOSFET, so the op-amp U1A has \SI{2.5}{\volt} at its non-inverting input. Assuming the op-amp is ideal (which means it has infinite gain and input impedance, and zero output impedance), the voltage at the inverting input will also be \SI{2.5}{\volt}. This means there is a constant \SI{2.5}{\volt} across R2, so it draws a constant \SI{11.4}{\milli\ampere}. Because the op-amp is assumed to be ideal, no current can go into the input pins, so the full \SI{11.4}{\milli\ampere} must be sourced from the op-amp output pin and hence pass through the LED D5. This creates a constant current source for the LED, assuming a constant \SI{2.5}{\volt} rail and that the ideal op-amp assumption is valid. When the MOSFET gate is driven high it turns on. This leads to the MOSFET having a very low impedance, so the non-inverting input is effectively tied to ground. Therefore R2 has no volts across it, and hence no current, so the LED turns off.\\

The MCP6002-I/SN \cite{mcp6002} was selected as the op-amp. This has a short circuit output current of \SI{23}{\milli\ampere}, so will comfortably be able to provide the desired \SI{11.4}{\milli\ampere}. Its input offset voltage is $\pm\SI{4.5}{\milli\volt}$, which leads to a potential current error of \SI{20}{\micro\ampere} (or 0.17\%), which is a negligible error. The input voltage noise density is \SI{28}{\nano\volt\per\sqrt{\hertz}} and the gain-bandwidth product is \SI{1}{\mega\hertz} so the worst case bandwidth is \SI{1}{\mega\hertz} (which assumes unity gain). This leads to an input noise voltage of \SI{28}{\micro\volt}, which again is negligible. The current source was tested for its invariance to temperature. When exposed to the heat fro ma hairdryer, the LED current only changed by 0.85\%, so there is confidence that the current source will be stable over the operating temperatures.\\

The light transmitted by the LED is then reflected off the target liver, and the received light is collected by phototransistor Q2. This produces a current proportional to the input light power. A transimpedance amplifier is used convert this to a readable voltage signal. Again assuming op-amp U1B is ideal implies that the voltage at the input pins are equal, and no current enters the pins. The phototransistor current $i_f$ therefore passes through the resistor R3, which leads to an output voltage signal $v_0 = 2.5 - i_fR_f$. The \SI{2.5}{\volt} rail at the non-inverting input acts to bias the op-amp, and was selected as half the voltage rails to ensure that the op-amp output signal could have maximum voltage swing before clipping at the supply rails.\\

\begin{table}[h]
	\centering
	\caption{Results for feedback resistor calibration experiments. The peak to peak signal is the difference between the output voltage when the LED was on and off, and the minimum signal is the output voltage when the LED was on.}
	\label{tab: tia feedback resistor}
	\begin{tabular}{|c|c|c|c|}
		\hline
		\textbf{LED Current} & \textbf{Feedback Resistor} & \textbf{Peak to Peak Signal} & \textbf{Minimum Signal}\\
		(mA)	&	(\si{\kilo\ohm})	&	(V)	&	(V)\\
		\hline
		\multirow{4}{*}{10}	&	2.2	&	0.93	&	1.20\\
						\cline{2-4}
						&	3.3	&	1.20	&	1.06\\
						\cline{2-4}
						&	4.7	&	1.60	&	0.62\\
						\cline{2-4}
						&	6.8	&	2.22	&	0.00\\
		\hline
		\multirow{4}{*}{20}	&	1.2	&	0.84	&	1.42\\
						\cline{2-4}
						&	1.5	&	1.07	&	1.15\\
						\cline{2-4}
						&	2.7	&	1.9	&	0.25\\
						\cline{2-4}
						&	3.3	&	2.09	&	0.00\\
		\hline
	\end{tabular}
\end{table}

The feedback resistor $R_f$ needed to be chosen to ensure good output characteristics. If it was too small, then the PIC's ADC would not be able to discriminate between changes in the light intensity.  The PIC16F688 uses a 10-bit ADC, so the maximum voltage resolution is \SI{4.88}{\milli\volt} when operated at \SI{5}{\volt}. The feedback resistor also cannot be too large, as this could lead to the op-amp saturating at the ground supply rail. Table \ref{tab: tia feedback resistor} shows the results of an experiment carried out to investigate the signal levels with different feedback resistors. The circuit was constructed on breadboard, with the LED and phototransistor soldered to stripboard to ensure they remained at a fixed distance of \SI{7.62}{\milli\metre}. This was then shone at a piece of A4 paper at a distance which led to the largest signal. The peak to peak signal is the difference between the transimpedance amplifier output voltage when the LED was on and off, and the minimum signal is the output voltage when the LED was on. Whenever the minimum signal was \SI{0}{\volt}, the amplifier had saturated so these values of $R_f$ are unsuitable. Therefore, the optimum $R_f = \SI{4.7}{\kilo\ohm}$, as this had the largest peak to peak signal without saturating. The system was tested using an LED current of \SI{20}{\milli\ampere} as well, but it was decided that this was too close to the op-amp output current limit to be a reliable, stable current source.\\

%LTSpice simulation

The transimpedance amplifier is a notoriously unstable circuit due to the phototransistor's junction capacitance. This can be stabilised by adding the capacitor C1 \cite{tia_stability}, which adds a zero into the feedback factor, compensating for the pole created by the phototransistor capacitance \cite{tia_stability}. The capacitor value was selected to be \SI{470}{\pico\farad}, as this gave a bandwidth of \SI{72}{\kilo\hertz} which is well above the required bandwidth of the circuit (which is \SI{10}{\kilo\hertz} at most), but is well below the gain-bandwidth product of \SI{1}{\mega\hertz} so it avoids the stability problems.\\

A button also needed to be designed for the probe, so the surgeon could indicate that they wanted to take a measurement. A standard push button could not be used because the remote unit had to be hermetically sealed. Therefore, a non-contact means of sensing a button press had to be devised. This could either be capacitive, sensing the finger like the touch screen on a mobile phone, or optical, by measuring light reflected back from the finger. It was found that the liver backscatter measurement system (figure \ref{fig: measurement schematic}) was discriminated well between the presence or absence of a finger at a range of a few centimetres, so this circuit was included twice within the remote unit to be used as a button. The downside to this method is that it is an active sensing system, requiring power to be supplied to the LED and phototransistor. This will limit the time the remote unit can operate over one full charge. \\



\subsection{Communications}
The Quasar QAM-TX3 \cite{qam-tx} and QAM-RX10-433 \cite{qam-rx} transmitter/receiver modules were used to implement the wireless link. These had all the amplification, demodulation, filtering, and other signal processing required for a radio link fully integrated. Fully integrated modules were selected because they were cheap and simplified the design greatly, as the design only had to focus on low frequencies and not the complex effects associated with RF electronics. The Quasar modules allowed the raw data to be input to the data pin in baseband, then modulated this onto a \SI{433}{\mega\hertz} carrier using on-off keying (OOK). The transmitter then sent this modulated signal to its antenna, and was received by the receiver which demodulated the signal and output the baseband signal at its data pin. The modules could work up to \SI{3}{\kilo\bit\per\second}, which is a suitable speed for the small packets of data which will need to be sent by the remote unit. They operate in the \SI{433}{\mega\hertz} band, which is an unlicensed ISM band in the UK \cite{ism_band} so there should be no regulatory issues using this frequency. A \SI{100}{\pico\farad} decoupling capacitor was used between the power rails of both modules, to smooth out any ripple in the power supply to ensure a reliable RF stage.\\

\subsection{Power and Wireless Charging}
The remote unit was designed to run from a \SI{5}{\volt} supply. \SI{3.3}{\volt} was not suitable as the LEDs had a forward voltage of \SI{1.5}{\volt}, so a \SI{3.3}{\volt} supply would potentially be too small for the current source designed in figure \ref{fig: measurement schematic}. Therefore, a set of batteries had to be selected which would provide this voltage. The batteries had to be small to ensure the overall remote unit design was small enough to fit in the surgeon's hand, and needed to be rechargeable because of the hermetic seal. Two different chemistries were available from RS Components \cite{rs} in a small, rechargeable package: lithium ion (Li-ion) and Nickel Metal Hydride (NiMH). The NiMH chemistry was more suitable due to the increased safety. Li-ion batteries are highly sensitive to over/under-voltage and high temperatures, and can blow up if these limits are exceeded \cite{batteries}. This requires complex protection circuitry and may mean that the battery will not survive being placed in an autoclave. NiMH are much more resilient to over and under charging, so are much safer than Li-ion. They also have a higher energy density \cite{batteries}. Nevertheless, they are lower voltage so more cells will be required to provide the desired \SI{5}{\volt}, have a long charging time, and they have a high self-discharge rate \cite{batteries}. This is not such an issue for this application, as the remote unit can remain charging until it is required for an operation, and then it will only be used for a short time before being placed on charge again. A \SI{2.4}{\volt} \SI{80}{\milli\ampere\hour} RS-Pro NiMH button cell \cite{rs_pro_batteries} was selected as suitable. Three of these were used to give a nominal \SI{7.2}{\volt} supply, which gave a margin of 44\% for when the cells started to discharge.\\

A power converter then needed to be selected to drop the \SI{7.2}{\volt} battery voltage to the required \SI{5}{\volt} rail. A Low-Dropout (LDO) voltage regulator was deemed suitable for this application, as it would provide a stable voltage rail. The advantage over a buck converter is the smaller device size and design simplicity, but this comes at the cost of worse efficiency. The MCP1702T-5002E/CB \cite{mcp1702} in a SOT-23A package was suitable. This was a \SI{5}{\volt} regulator with a rated output current of \SI{250}{\milli\ampere}. This was well above the \SI{50}{\milli\ampere} maximum current required by the remote unit (see section \ref{power budget}). The quiescent current was \SI{2}{\micro\ampere}, which is negligible compared with the remainder of the circuit, so the regulator will not affect the battery lifetime. The regulator will dissipate a maximum of $\SIrange{7.2}{5}{\volt}\times\SI{50}{\milli\ampere}=\SI{110}{\milli\watt}$. The SOT-23A package has a junction to air thermal resistance of \SI{336}{\celsius\per\watt}, so the maximum temperature increase is \SI{37}{\celsius}. In fact, the device will rarely operate at \SI{50}{\milli\ampere}, so the regulator will remain at ambient temperature.\\




Inductive charging was selected as an appropriate 

%2.5V rail
%Battery voltage sensor
%Battery current sensor

\subsection{Remote Unit}
%Temperature sensor
%PicKit programmer
%Power budget

\subsection{Base Unit}
%Memory 
%RTC
%LCD
%switches
%power budget
%
%
%\section{PCB Design}
The PCB designs were produced using KiCAD \cite{kicad}. This is an integrated CAD package which allows schematics, PCB layouts and PCB footprints to be designed within one project. The PCBs were manufactured by an external company, JLCPCB \cite{pcb_house}, which produced high-quality boards for low cost. The first step in the PCB design was to specify the design rules for the PCB layout according to the tolerances set by JLCPCB. These included restrictions such as the minimum track width, hole size and clearances allowed. A 2-layer design was selected so that none of the traces ran through an inaccessible middle layer, which meant debugging would be easier. Furthermore, it meant that if any errors were made in the design, it would be easier to fix them as the traces could be cut and new connections soldered on. After the traces had been routed, the remaining board area was filled with power planes. The \SI{+5}{\volt} plane was on the top layer, and the ground plane was on the bottom layer. Using power planes was advantageous as it lowered the resistance of the path from the power node to the components, so there was a smaller voltage drop across the trace. Furthermore, the copper planes act to shield the circuit from electromagnetic interference and also stop any emissions from the board. Finally, using wide planes helps to lower the loop inductance, so there is less likelihood of ringing or voltage spikes. The other traces on the board were made as wide as possible to lower the resistance and make them easier to modify if any errors were made. Once the designs were produced, Gerber and drill files were generated, and these were sent to the board house for manufacture.\\


\subsection{Remote Unit}
The main restriction on the remote unit was that it had to be handheld, meaning the PCB should be made as small as possible. For this reason, many of the components were made surface mount, as generally these had a smaller footprint and did not waste board space with through-holes. Standard 0805 packages were used for the resistors and capacitors, and SOIC packages were used for the op-amps and the PIC.\\

\begin{figure}[htb]
	\centering
	\includegraphics[width=0.8\linewidth]{old remote pcb.png}
	\caption{First version of the remote unit PCB.}
	\label{fig: old remote pcb}
\end{figure}

A first iteration of the design was manufactured (figure \ref{fig: old remote pcb}). This was useful for testing the basic functionality of the circuit and highlighted several issues which were resolved in the second iteration. Several additional test points and sockets for receiver modules were included for testing purposes, and these were removed from the final design. Additionally, an error was made in the differential amplifier circuits as they were connected to \SI{0}{\volt} rather than the \SI{2.5}{\volt} ``pseudo-ground'' rail. \\
%Take out mistake about 0V rail





Figure \ref{fig: remote pcb} shows the revised PCB design. In addition to removing many test points, the positions of the LEDs and phototransistors were changed to be more reflective of their intended positions in the probe. They were soldered to surface mount pads to ensure that they pointed in the correct direction and were accurately aligned to the front/back of the probe. They were covered with black electrical tape to block any cross-talk, which ensured that the signals measured were solely due to the reflectance of the object they were pointed at. Additionally, the battery footprints were changed to use slots rather than circular holes, as these were a better fit for the battery pins. This design measured \SI{53x38}{\milli\metre}, which was small enough to fit in a hand.
\begin{figure}[htb]
	\centering
	\includegraphics[width=0.8\linewidth]{remote pcb.png}
	\caption{Final version of the remote unit PCB.}
	\label{fig: remote pcb}
\end{figure}




\FloatBarrier
\subsection{Base Unit}
\begin{figure}[h!]
	\centering
	\includegraphics[width=0.8\linewidth]{base pcb.png}
	\caption{Base unit PCB.}
	\label{fig: base pcb}
\end{figure}

The shape of the base unit was designed to fit within the selected enclosure (see section \ref{base casing}). The final design is shown in figure \ref{fig: base pcb}. There was a lot more space on the base unit so more through-hole components could be used. Specifically, some of the power components needed a through-hole package because the thermal rating of the surface mount alternatives was too small. The PIC also had a through-hole package so that it could be easily transferred to a breadboard for testing. Many components were connected via header sockets, which allowed flexibility during testing. In the commercial version, these components should be mounted directly on the PCB to ensure they do not accidentally disconnect. Any components which cannot be soldered directly should use terminal blocks rather than header sockets to make a stronger connection. 
%
%
%%\section{Firmware Design}
%Config bits
%TRIS/ANSEL bits

\subsection{Measurements Algorithm}


\subsection{Communications Algorithm}

\subsection{Wireless Charging Algorithm}
The wireless charger requires a PWM signal with frequency $f=\SI{100}{\kilo\hertz}$ and duty cycle $D=0.5$. The PIC18F2550 has a built-in Capture/Compare/PWM (CCP) module \cite{pic18f2550}. This allows the PWM signal to be generated in hardware, rather than having to worry about precisely timing the software. A software implementation would have been difficult, as the PIC must do many other tasks as well as the PWM signal generation (such as listening for any incoming messages on the receiver), which would result in a variable delay time before the PWM could next be toggled. This is clearly undesirable, as it would not produce the desired constant frequency and duty ratio signal. The CCP pin could be set up as PWM at the beginning of the program and would continue to generate the signal regardless of what the rest of the program was doing.\\

PICBASIC contains a built-in command to access the CCP pins, HPWM (hardware PWM). This allows PWM to be set up simply by specifying the output pin, frequency and duty cycle \cite{picbasic pro}. However, because the free version of MicroCode Studio does not support the long data type, the highest frequency it can deliver is \SI{32.767}{\kilo\hertz}. The wireless charger was designed to run at \SI{100}{\kilo\hertz}, so the PWM signal needed to be faster than HPWM would allow. Therefore, a series of register writes were programmed to set up the CCP pin. The hardware was designed to use CCP module 2 on pin 24, which was multiplexed to PORTB.3. CCP2 could be multiplexed to either PORTC.1 (the default) or PORTB.3. Therefore, the appropriate CONFIG bit was set. In PICBASIC, this was \verb|CONFIG CCP2MX = OFF|.\\

The PIC18F2550 data sheet \cite{pic18f2550} detailed the required register writes to set up CCP2 as a PWM pin. First, the PWM period was written to the PR2 register (Timer2 period register). The PWM period is given by the equation:

\begin{equation}
\text{PWM Period} = [\text{PR2}+1] \times 4 \times T_\text{osc} \times (\text{TMR2 Prescale})
\end{equation}

where $T_\text{osc}$ is the oscillator period, so for a \SI{8}{\mega\hertz} clock speed $T_\text{osc}=\SI{125}{\nano\second}$. The factor of 4 is because each PIC operation takes four clock cycles to be executed, so the effective clock speed is $f_\text{osc}/4$. For a TMR2 prescaler setting of x1, the required PR2 value was 19 for a \SI{100}{\kilo\hertz} PWM signal. Every clock cycle, the Timer2 module increments the TMR2 register. TMR2 is then compared to the PR2 register, and if the two are equal then the TMR2 is reset to 0 on the next clock cycle \cite{pic18f2550}. This is how the \SI{100}{\kilo\hertz} interval is generated. The duty cycle was then defined by the equation:\\

\begin{equation}
\text{PWM Duty Cycle} = (\text{CCPR2L:CCP2CON\textless5:4\textgreater}) \times T_\text{osc} \times (\text{TMR2 Prescale Value}) \label{eq: duty cycle}
\end{equation}

Where CCPR2L is the CCP2 register low byte and CCP2CON\textless5:4\textgreater are bits 5 and 4 of the CCP2 control register which define the lower two bits of the duty cycle period. Equation \ref{eq: duty cycle} $(\text{CCPR2L:CCP2CON\textless5:4\textgreater}) = \num{4e6}$, so 0x0A was written to CCPR2L and 00 was written to CCP2CON\textless5:4\textgreater. After this was done, TRISB.3 was cleared to set the pin to a digital output. The TMR2 prescaler was set to x1 and Timer2 was enabled by writing 0x04 to T2CON (Timer2 control register). Finally, CCP2CON\textless3:2\textgreater was set to 11, to set CCP2 to PWM mode. This generated a stable \SI{100}{\kilo\hertz} square wave.\\
%
%
%%\section{System Evaluation}
%%\subsection{Measurement System}
%%\subsection{Power}
%%\subsection{Communications}
%%\subsection{Overall System}
%
%
%\section{Casing Design}

\subsection{Remote Unit}
The remote unit would have the PCB enclosed in a resin cast, which would provide a hermetic seal. The resin must be a biocompatible material to ensure the device is safe to use in vivo and that it will be approved by the regulator. There are lots of adverse affects an improper material could have on a patient depending on the chemical and physical properties of the material, so extensive testing must be performed on a material to determine if it is safe to use in contact with the body \cite{biocompatible_tests}. Luckily, there are many commercially available biocompatible resins, as they are commonly used in fields like dentistry which commonly uses PMMA for casts \cite{biocompatible_resin}. The resin cast needs to have two transparent ``windows'' built in on the two ends, to allow light to pass to and from the LEDs and phototransistors. If a transparent resin is used, this will not be a problem. If a opaque resin is used, then a transparent glass screen should be included to allow the transmission of light. This will also result in an air gap needed around the LEDs and phototransistors.\\

\begin{figure}[htbp]
	\centering
	\begin{subfigure}[b]{0.4\linewidth}
		\includegraphics[width=\linewidth]{cuboid.png}
		\caption{Cuboid}
		\label{fig: cuboid}
	\end{subfigure}
	\begin{subfigure}[b]{0.4\linewidth}
		\includegraphics[width=\linewidth]{cylinder.png}
		\caption{Cylinder}
		\label{fig: cylinder}
	\end{subfigure}
	\begin{subfigure}[b]{0.4\linewidth}
		\includegraphics[width=\linewidth]{half cylinder.png}
		\caption{Half Cylinder}
		\label{fig: half cylinder}
	\end{subfigure}
	\begin{subfigure}[b]{0.4\linewidth}
		\includegraphics[width=\linewidth]{final design.png}
		\caption{Final Design}
		\label{fig: final design}
	\end{subfigure}
	\caption{3D renders of potential casing designs produced using SketchUp \cite{sketchup}}
	\label{fig: casings}
\end{figure}

The design of the casing shape underwent several iterations to improve the user experience. The iterations were rendered in CAD (see figure \ref{fig: casings}) and modelled using Play-Doh to see how they felt to hold. The first design was a simple cuboid shape (figure \ref{fig: cuboid}), with dimensions large enough to enclose the PCB (which measured \SI{80x40x35}{\milli\metre}). This felt cumbersome to hold, and the corners were uncomfortable. The design was therefore changed to a cylinder (figure \ref{fig: cylinder}), to make the edges smooth. This did feel more comfortable, but the cylindrical shape could lead to issues with the wireless charger. Maximum power is transferred when the charging coils are parallel, and zero power is transferred when they are perpendicular. Due to the rotational symmetry of the cylinder, it will not be clear which orientation the surgeon should place the remote unit on the charging pad, and it could possibly roll over and misalign. Thus, a new half cylinder design (figure \ref{fig: half cylinder}) was made. The flat surface on the bottom makes it clear which side should be placed against the charging pad. Furthermore, if the half cylinder is placed on its curved side, gravity will rotate the device so the flat surface is pointing upwards, so the coils will still be parallel albeit at a greater separation. One problem noted with this design (and all the previous ones) was that it was not clear which side of the device should be placed against the liver and which side has the ``take measurement'' button. Therefore, the final design (figure \ref{fig: final design}) included an extrusion on the front of the device which should be held against the liver. The device should be held like a syringe, and the surgeon uses their thumb to press the button on the back. It was also noted that in order to house the PCB, the half cylinder design needed to have a diameter of \SI{70}{\milli\metre}, which was too big. The final design used a more circular shape to reduce the overall radius, but still included a flat edge on the bottom.\\

%REAL LIFE MODEL
%
%\section{Overall System}
\begin{figure}[htbp]
	\centering
	\includegraphics[width=\linewidth]{flowchart.png}
	\caption{Flowchart for base unit firmware. Blue blocks indicate states, and green blocks indicate processes executed within these states.}
	\label{fig: flowchart}
\end{figure}

\begin{figure}[htbp]
	\centering
	\begin{subfigure}[b]{0.4\linewidth}
		\includegraphics[width=\linewidth]{menu screen.jpg}
		\caption{Menu screen}
		\label{fig: menu screen}
	\end{subfigure}
	\begin{subfigure}[b]{0.4\linewidth}
		\includegraphics[width=\linewidth]{take readings.jpg}
		\caption{Take readings screen}
		\label{fig: take readings screen}
	\end{subfigure}
	\begin{subfigure}[b]{0.4\linewidth}
		\includegraphics[width=\linewidth]{view readings.jpg}
		\caption{View readings screen}
		\label{fig: view readings screen}
	\end{subfigure}
	\begin{subfigure}[b]{0.4\linewidth}
		\includegraphics[width=\linewidth]{delete screen.jpg}
		\caption{Delete readings screen}
		\label{fig: delete readings screen}
	\end{subfigure}
	\caption{LCD screens for all four system states.}
	\label{fig: lcd screens}
\end{figure}

Figure \ref{fig: flowchart} shows the flowchart for the base unit's software. It has four states: ``menu'', ``take readings'', ``view readings, and ``delete readings''. Figure \ref{fig: lcd screens} shows what is displayed on the LCD for each state. The system boots up in the menu state, where the user navigates between the options by using the up and down buttons, and selects an option by pressing the OK button. Once an option is selected, the system moves to that state. The system will transition back to the menu state whenever the menu button is pressed, regardless of what the system is currently doing, because the menu button is connected to the $\overline{\text{MCLR}}$ hardware reset.\\

In the ``take readings'' state, the system listens for data from the remote unit, and when it picks up a valid data packet it will display it on the LCD. The system will compare the reading to a preset threshold for fatty/lean livers, and display this to the user. It will also display the current time and the remote unit's temperature.\\

In the ``view readings'' state, the system will initially display the two most recent data points stored in EEPROM along with the date and time that they were taken. By using the up and down buttons, the user can scroll through the readings stored in EEPROM.\\

In the ``delete readings'' state, the user is first asked to confirm that they do intend to delete all the readings stored in memory. If they press the OK button, then all the readings in the EEPROM will be cleared. If they press any other button, then the system will move back to the menu screen and not delete the readings.\\

The total price of the remote unit PCB and components was £17.49, and the price of the base unit was £33.66. These prices exclude the cost of resistors, capacitors and wire as these were already available. The remote unit price is well below the £45 price requirement, so there is confidence that the NHS will see it as cost effective. The base is a little more expensive, but this is allowable since it will be reused and fewer base units will need to be bought. The base is much more expensive because of the £11.67 LCD and more expensive £6.08 PIC18F2550 microcontroller. The cost of the enclosure (£9.46) must also be added, leading to a final overall cost of £60.61. The price may be further reduced when the system moves to mass production, as this will allow cost savings when components are bought in bulk. Furthermore, the major cost of the PCB manufacture was the shipping (for instance the remote unit PCB cost £0.28 per unit but £3.78 for shipping). Clearly, when large batch numbers of PCBs are made, the shipping cost per unit will reduce so they also become cheaper per unit.\\

%
%\section{Conclusion}
The final system met many of the requirements set out at the start. The system consists of two parts, a remote unit and base unit. The remote unit is small, handheld, and has an easy to understand interface. This means it will be easy for surgeons to use it in clinic. The results it produces are very well correlated to the original probe, so there is confidence that it should perform well in clinical trials.The remote unit is charged wirelessly, with a maximum output current of \SI{37}{\milli\ampere} which can be reduced with frequency modulation. It communicates wirelessly with the base unit over a maximum reliable range of \SI{00000000}{\metre}, which will cover the length of the operating theatre. Detailed designs have been produced for the remote unit's casing, which will be cast from a biocompatible resin. The cast will provide a hermetic seal and mechanical strength. The base unit contains features such as an LCD screen, EEPROM to store readings and an RTC to timestamp data. The data is output to a PC over an RS232 link.\\

The PCB designs included several test points and current sense resistors which can be removed for the final design. Furthermore, many components were designed to plug-in to header sockets, but these could be changed to solder on packages or terminal blocks to ensure reliability in the commercial release. Throughout the project, time was spent on developing skills such as PIC programming and debugging. Furthermore, a greater intuition into where problems within a circuit originate was developed, as well as the knowledge of what to do to fix them. If the project were completed again, then it could progress quicker as these skills would not need to be re-learned. \\

The next stages of the project is to construct the casings for the system when labs re-open after COVID-19. The system can then undergo a rigorous series of laboratory and clinical tests to evaluate its performance as predicting transplant survival. The \SI{5}{\volt} regulator may be changed to a buck-boost converter to maximise the battery lifetime.\\
%
%
%
%
%
%\newpage
%
%
%\bibliography{bibliography}
%
%\newpage
%\appendix
%\appendixpage
%\addappheadtotoc
%
%\section{Risk Assessment Retrospective}
%The risk assessment was followed without many issues. The main risks were soldering and sharp knives, which were anticipated and did not result in injury because all steps in the risk assessment were followed. When labs were closed in Lent term due to COVID-19, soldering had to be done in my college room. A portable fume extractor was used and the window was left open for ventilation, leading to minimal fume inhalation. Safety goggles were always used when soldering.\\
%
%One risk which was not anticipated was burns from hot components. When a component was connected wrongly, it could become very hot which led to the occasional burn. Once this issue was noted, extra care was taken when touching suspect components, and any small burns which did happen were quickly taken care of as I had adequate first aid training. This risk should be included in any future risk assessments associated with this project.\\
%
%\newpage
%\section{Remote Unit Circuit Schematic}
%\begin{figure}[h!]
%	\centering
%	\includegraphics[width=\linewidth]{remote schematic.png}
%\end{figure}
%
%\newpage
%\section{Base Unit Circuit Schematic}
%\begin{figure}[h!]
%	\centering
%	\includegraphics[width=\linewidth]{base schematic.png}
%\end{figure}
%
%
%
%\newpage
%\section{COVID Impact Statement}
%The main impact COVID-19 had on my project was delays sourcing equipment and components. I could not begin any practical work before November 2nd because I had to wait for my breadboarding equipment and PICKit programmer to be purchased from the suppliers and then posted to me. Other orders also suffered delays due to supplier or postage issues.\\
%
%Further delays were incurred with PCB manufacture, as CUED's rapid PCB prototyping facilities were unavailable, meaning boards had to be purchased from a 3rd party company based in China. Furthermore, I did not have access to equipment to solder SMD parts, so I had to rely on my supervisor to do these for me. All in all, this resulted in a lead time of 3-4 weeks to receive a PCB from when I sent away the designs. These delays slowed down my project, and meant the final testing could not start before the Easter vacation (I received the boards on April 8th after sending them for manufacture on March 5th).\\
%
%Access to labs was stopped after Michaelmas term, so I had to rely on equipment available at home to carry out testing. Luckily the project was fairly small so everything could be done on my desk (as specified in the COVID-19 contingency plan). However, I did not have access to accurate laboratory measuring equipment, so the quality of the data may not be the best and measurements should be repeated under laboratory conditions. The other downside to not having lab access was that I was unable to cast the remote unit in a resin like planned, so the product lacks its final casing.\\
%
%
%
%
%Unable to cut box

\end{document}
