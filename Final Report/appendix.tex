\section{Risk Assessment Retrospective}
The risk assessment was followed without many issues. The main risks were soldering and sharp knives, which were anticipated and did not result in injury because all steps in the risk assessment were followed. When labs were closed in Lent term due to COVID-19, soldering had to be done in my college room. A portable fume extractor was used, and the window was left open for ventilation, leading to minimal fume inhalation. Safety goggles were always used when soldering.\\

One risk which was not anticipated was burns from hot components. When a component was connected wrongly, it could become very hot which led to the occasional burn. Once this issue was noted, extra care was taken when touching suspect components, and any small burns which did happen were quickly taken care of as I had adequate first aid training. This risk should be included in any future risk assessments associated with this project.\\

\newpage
\section{Remote Unit Circuit Schematic}
\begin{figure}[h!]
	\centering
	\includegraphics[width=\linewidth]{remote schematic.png}
\end{figure}

\newpage
\section{Base Unit Circuit Schematic}
\begin{figure}[h!]
	\centering
	\includegraphics[width=\linewidth]{base schematic.png}
\end{figure}



\newpage
\section{COVID Impact Statement}
The main impact COVID-19 had on my project was delays sourcing equipment and components. I could not begin any practical work before November 2nd because I had to wait for my breadboarding equipment and PICkit programmer to be purchased from the suppliers and then posted to me. Other orders also suffered delays due to supplier or postage issues.\\

Further delays were incurred with PCB manufacture, as CUED's rapid PCB prototyping facilities were unavailable, meaning boards had to be purchased from a 3rd party company based in China. Furthermore, I did not have access to equipment to solder SMD parts, so I had to rely on my supervisor to do these for me. All in all, this resulted in a lead time of 3-4 weeks to receive a PCB from when I sent away the designs. These delays slowed down my project, and meant the final testing could not start before the Easter vacation (I received the boards on April 8th after sending them for manufacture on March 5th).\\

Access to labs was stopped after Michaelmas term, so I had to rely on equipment available at home to carry out testing. Luckily, the project was fairly small so everything could be done on my desk (as specified in the COVID-19 contingency plan). However, I did not have access to accurate laboratory measuring equipment, so the quality of the data may not be the best and measurements should be repeated under laboratory conditions. The other downside to not having lab access was that I was unable to cast the remote unit in a resin like planned, so the product lacks its final casing.\\




Unable to cut box