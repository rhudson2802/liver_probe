\section{Conclusion}
The final system met many of the requirements set out at the start. The system consists of two parts, a remote unit and base unit. The remote unit is small, handheld, and has an easy-to-understand interface, which will make it easy for surgeons to use in clinic. The results it produces are strongly correlated to the original probe, so there is confidence that it should perform well in clinical trials. The remote unit is charged wirelessly, with a maximum output current of \SI{37}{\milli\ampere} which can be reduced with frequency modulation. It communicates wirelessly with the base unit over a maximum reliable range of \SI{14}{\metre}, which will cover the length of the operating theatre. Detailed designs have been produced for the remote unit's casing, which will be cast from a biocompatible resin. The cast will provide a hermetic seal and mechanical strength. The base unit contains features such as an LCD screen, EEPROM to store readings and an RTC to timestamp data. The data is output to a PC over an RS232 link.\\

The PCB designs included several test points and current sense resistors which can be removed for the final design. Furthermore, many components were designed to plug-in to header sockets, but these could be changed to solder on packages or terminal blocks to ensure reliability in the commercial release. The \SI{5}{\volt} regulator may be changed to a buck-boost converter to maximise the battery lifetime. Throughout the project, time was spent on developing skills such as PIC programming and debugging. Furthermore, a greater intuition into problem solving in practical electronics was developed. If the project were completed again, then it could progress quicker as these skills would not need to be re-learned. \\

The next stage of the project is to construct the casings for the system when labs re-open after COVID-19. The system can then undergo a rigorous series of laboratory and clinical tests to evaluate its performance at predicting transplant survival. Feedback can also be gathered from the surgeons who tested it, to make sure the probe was easy to use and that it is likely to get adapted for full-scale clinical use.