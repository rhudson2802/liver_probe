\section{Conclusion and Evaluation}
The final system met many of the requirements set out at the start. The system consists of two parts: a remote unit and a base unit. The remote unit is small, handheld, and has an easy-to-understand interface, which will make it easy for surgeons to use in clinic. The results it produces are strongly correlated to the original probe, so there is confidence that it should perform well in clinical trials. The remote unit is charged wirelessly, with a maximum output current of \SI{37}{\milli\ampere}, which can be reduced with frequency modulation. This will fully charge the unit in 2 hours 10 minutes. It communicates wirelessly with the base unit over a maximum reliable range of \SI{14}{\metre}, which will cover the length of the operating theatre. Detailed designs have been produced for the remote unit casing, which will be cast from a biocompatible resin. The cast will provide a hermetic seal and mechanical strength, which were important to ensure safety and reliability. The base unit contains features such as an LCD screen, EEPROM to store readings, and an RTC to timestamp the data. The data is output to a PC over an RS232 link.\\

%The PCB designs included several test points and current sense resistors which can be removed from the final design. Furthermore, many components were designed to plug into header sockets, but these could be changed to solder-on packages or terminal blocks to ensure reliability in the commercial release.

Most of the issues encountered in the project regarded the design of the power systems. During testing, the battery voltage reduced a lot when loaded, leading to the \SI{5}{\volt} regulator output dropping below \SI{5}{\volt}. If the project were done again, the regulator may be changed to a buck-boost converter to maximise the battery lifetime, as this would be able to provide a stable output whether the input was above or below \SI{5}{\volt}. Furthermore many problems were encountered attempting to deliver a large enough charging current to the batteries. The inclusion of a magnetic core in the primary coil may aid the flux distribution and lead to a greater flux linkage, improving the power transferred. A more complex wireless charging system with feedback (such as Qi) may have led to improved efficiency, but was not possible within the time constraints of the project.\\

Another area for improvement was the \verb|serin2| and \verb|serout2| commands used for communications, which were blocking. This restricted what else the program could do while it was waiting for data. This could have been avoided by either writing a bespoke routine (which would have been complicated and took a long time) or changing language from PICBASIC to C (as C has many more libraries available). Changing to C would also allow a greater choice in the PIC used, as it would not be limited to the selection MicroCode studio could compile for free. Nevertheless, PICBASIC was still a good choice of language due to its simplicity and familiarity to the developers.\\

Additionally, if the labs were open, then the subsystems could have been prototyped using CUED's rapid PCB manufacturing facilities, which would have made testing the surface mount components easier and meant more designs could be iterated without the long delays associated with ordering from a third-party company. This would also have helped flag the mistakes which were made in the PCBs before a final iteration was produced. \\

Throughout the project, time was spent on developing skills such as PIC programming and debugging. Furthermore, a greater intuition into problem-solving in practical electronics was developed. For instance, some components were damaged by electrostatic discharge or too large currents, but it took a long time to deduce that the component was broken rather than the circuit design. If the project were completed again, then it could progress quicker as these skills would not need to be re-learned. \\

The next stage of the project is to construct the casings for the system when labs re-open after COVID-19. The system can then undergo a series of rigorous laboratory and clinical tests to evaluate its performance in predicting transplant survival. Feedback can also be gathered from the surgeons who tested it, to make sure that the probe is easy to use and is likely to get adapted for full-scale clinical use.\\