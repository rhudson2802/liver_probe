\section{Measurement System}
\subsection{Hardware}

\begin{figure}[htb]
	\centering
	\includegraphics[width=\linewidth]{measurement.PNG}
	\caption{Circuit schematic of backscatter measurement system.}
	\label{fig: measurement schematic}
\end{figure}

Figure \ref{fig: measurement schematic} illustrates the circuit used to perform the backscatter measurements on the liver. The clinical trials of the original probe showed that the neither the LED wavelength nor the LED-photodetector spacing affected the correlation with PNF, so this was not a critical design requirement of the new probe. Therefore, an LED in the near infra-red (NIR) range was selected (Vishay TSHA4401 \SI{875}{\nano\metre} \cite{tsha4401}) along with a phototransistor also in the NIR range (Vishay BPW85B \SI{850}{\nano\metre} \cite{bpw85b}).\\

To ensure repeatable readings, the LED had to be supplied with a constant current, as the brightness of an LED is proportional to its current. The circuit on the left of figure \ref{fig: measurement schematic} was used to provide a constant current. When MOSFET Q1 was off, the voltage at the non-inverting input of the op-amp U1A was \SI{2.5}{\volt}. By the ideal op-amp assumption, the voltage at the inverting input was also \SI{2.5}{\volt}. This fixed the voltage across resistor R2 to \SI{2.5}{\volt}, so it drew \SI{11.4}{\milli\ampere} of current. As no current went through the op-amp input, all the current must have come from the op-amp output and have went through the LED.  This ensured a constant current through the LED. When the MOSFET was on, it pulled the non-inverting input to ground. Therefore, R2 had no volts across it, and hence drew no current, so the LED turns off.\\

%To ensure repeatable readings the LED had to be supplied with a constant current, as the brightness of an LED is proportional to its current. The circuit on the left of figure \ref{fig: measurement schematic} was used to provide a constant current. When MOSFET Q1 is off, the voltage at the non-inverting input of the op-amp U1A is \SI{2.5}{\volt}, since the op-amp and MOSFET draw negligible current. Therefore, the voltage at the inverting input is also \SI{2.5}{\volt}, as the input voltages of an ideal op-amp are equal. This fixes the voltage across resistor R2 to \SI{2.5}{\volt}, making it draw \SI{11.4}{\milli\ampere} of current. As no current goes to the inverting input of the op-amp, all the current must come from the op-amp output and go through the LED. This ensures a constant current through the LED. When the MOSFET gate is driven high it turns on. This leads to the MOSFET having a very low impedance, so the non-inverting input is effectively tied to ground. Therefore, R2 has no volts across it, and hence no current, so the LED turns off.\\
%This can be cut down

%To ensure repeatable readings the LED had to be supplied with a constant current, as the brightness of an LED is proportional to its current. The circuit on the left of figure \ref{fig: measurement schematic} was used to provide a constant current. First, consider when MOSFET Q1 is off. The voltage at the non-inverting input of the op-amp U1A is \SI{2.5}{\volt}, as the op-amp and MOSFET draw negligible current. Therefore, the voltage at the inverting input is also \SI{2.5}{\volt}, as the input voltages of an ideal op-amp are equal. This fixes the voltage across resistor R2 to \SI{2.5}{\volt}, making it draw \SI{11.4}{\milli\ampere} of current. As no current goes to the inverting input of the op-amp, all the current must come from the op-amp output and go through the LED. This ensures a constant current through the LED. When the MOSFET gate is driven high it turns on. This leads to the MOSFET having a very low impedance, so the non-inverting input is effectively tied to ground. Therefore, R2 has no volts across it, and hence no current, so the LED turns off.\\

The MCP6002-I/SN \cite{mcp6002} was selected as the op-amp. This had a short circuit output current of \SI{23}{\milli\ampere}, so was comfortably able to provide the desired \SI{11.4}{\milli\ampere}. Its input offset voltage was $\pm\SI{4.5}{\milli\volt}$, which led to a potential current error of \SI{20}{\micro\ampere} (or 0.17\%), which is negligible. The input voltage noise density was \SI{28}{\nano\volt\per\sqrt{\hertz}}, which led to a maximum input noise voltage of \SI{28}{\micro\volt}, which is again negligible. When exposed to the heat from a hairdryer, the LED current only changed by 0.85\%, so there is confidence that the current source will be stable over the operating temperatures.\\

The light transmitted by the LED was then reflected off the target liver, and the received light was collected by phototransistor Q2. This produced a current proportional to the input light power. A transimpedance amplifier was used convert this to a readable voltage signal. Assuming op-amp U1B is ideal, the phototransistor current $i_f$ passes through the resistor R3, which led to an output voltage signal $v_0 = 2.5 - i_f R_f$. The \SI{2.5}{\volt} rail at the non-inverting input acted to bias the op-amp. This was selected as half the supply rails to ensure that the op-amp output signal could have maximum voltage swing before clipping at the supply rails.\\

\begin{table}[htb]
	\centering
	\caption{Results for feedback resistor calibration experiments. The peak-to-peak signal is the difference between the output voltage when the LED was on/off, and the minimum signal is the output voltage when the LED was on.}
	\label{tab: tia feedback resistor}
	\begin{tabular}{|c|c|c|c|}
		\hline
		\textbf{LED Current} & \textbf{Feedback Resistor} & \textbf{Peak to Peak Signal} & \textbf{Minimum Signal}\\
		(mA)	&	(\si{\kilo\ohm})	&	(V)	&	(V)\\
		\hline
		\multirow{4}{*}{10}	&	2.2	&	0.93	&	1.20\\
						\cline{2-4}
						&	3.3	&	1.20	&	1.06\\
						\cline{2-4}
						&	4.7	&	1.60	&	0.62\\
						\cline{2-4}
						&	6.8	&	2.22	&	0.00\\
		\hline
		\multirow{4}{*}{20}	&	1.2	&	0.84	&	1.42\\
						\cline{2-4}
						&	1.5	&	1.07	&	1.15\\
						\cline{2-4}
						&	2.7	&	1.9	&	0.25\\
						\cline{2-4}
						&	3.3	&	2.09	&	0.00\\
		\hline
	\end{tabular}
\end{table}

The feedback resistor $R_f$ needed to be chosen to ensure good output characteristics. If it was too small, then the PIC's ADC would not be able to discriminate between changes in light intensity.  The PIC16F688 uses a 10-bit ADC, so the maximum voltage resolution is \SI{4.88}{\milli\volt} when operated at \SI{5}{\volt}. The feedback resistor also cannot be too large, as this could lead to the op-amp saturating at the ground supply rail. Table \ref{tab: tia feedback resistor} shows the results of an experiment carried out to investigate the signal levels with different feedback resistors. The LED and phototransistor were soldered to stripboard to ensure  a fixed separation of \SI{7.62}{\milli\metre}. This was shone at a piece of white printer paper at a distance which led to the largest signal. The optimum $R_f = \SI{4.7}{\kilo\ohm}$, as this had the largest peak to peak signal without saturating. The system was tested using an LED current of \SI{20}{\milli\ampere} as well, but it was decided that this was too close to the op-amp output current limit to be a reliable current source.\\

The transimpedance amplifier is a notoriously unstable circuit due to the phototransistor's junction capacitance. This can be stabilised by adding the capacitor C1, which compensates for the pole created by the phototransistor capacitance \cite{tia_stability}. The capacitor value was selected to be \SI{470}{\pico\farad}, as this gave a bandwidth of \SI{72}{\kilo\hertz}, which is well above the required bandwidth of the circuit (which is \SI{10}{\kilo\hertz} at most). It is also below the gain-bandwidth product of \SI{1}{\mega\hertz} so will avoid the stability problems.\\

A button also needed to be designed for the probe, so the surgeon could indicate that they wanted to take a measurement. A standard push button could not be used because the remote unit had to be hermetically sealed. Therefore, a non-contact means of sensing a button press had to be devised. This could either be capacitive, sensing the finger like the touch screen on a mobile phone, or optical, by measuring light reflected from the finger. It was found that the liver backscatter measurement system (figure \ref{fig: measurement schematic}) discriminated well between the presence or absence of a finger at a range of a few centimetres, so this circuit was used as the button. The downside to this method is that it is an active sensing system, requiring power to be supplied to the LED and phototransistor.\\

\begin{figure}[htb]
	\centering
	\includegraphics[width=0.35\linewidth]{2-5v rail.PNG}
	\caption{Op-amp unity buffer used to generate \SI{2.5}{\volt} rail.}
	\label{fig: 2.5v rail}
\end{figure}

The measurement system used a \SI{2.5}{\volt} reference rail. This rail had to be stable as it provided the reference signal for the LED current source and transimpedance amplifier, so any variations in the \SI{2.5}{\volt} signal would lead to measurement errors. An op-amp unity buffer circuit (figure \ref{fig: 2.5v rail}) was used to generate this signal from a potential divider reference. This is more stable than directly using a potential divider, as any loading of the potential divider will decrease the reference voltage it provides. Because an op-amp draws negligible current, the resulting output voltage will be much more stable. The \SI{2.5}{\volt} rail is only used to supply current to a \SI{10}{\kilo\ohm} resistor (\SI{250}{\micro\ampere}) and op-amp input (negligible current), so the buffer will not have to provide significant current and will remain stable.\\




\subsection{Firmware}
The system would give different readings depending on the background lighting conditions, as these would appear as an offset in the phototransistor current. To add resilience to background light, each measurement was the difference between the phototransistor reading when the LED was on and when it was off. Assuming the phototransistor has a linear incident light intensity-current response, this difference will leave just the information about the liver with no dependence on background light. A series of readings was made (N=50) and the average taken, to further increase the resilience to noise. The PICBASIC code is given below.

\begin{lstlisting}
ir_led var PORTC.2
red_led var PORTC.0
phototransistor con 7       '' Analogue port used as input

reading var word
dark var word
light var word
N var byte
i var byte

mainloop:
reading = 0
dark = 0
light = 0

low red_led

for i = 1 to N
    low ir_led					'' Turn on IR LED
    pause 1
    adcin phototransistor, light		'' read in light data
    high ir_led					'' Turn off IR LED
    pause 1
    adcin phototransistor, dark		'' read in dark data
    
    '' Add current readings to sum
    '' Dark should be > light if measurement variation is due to our probe
    '' Any readings with light > dark are noise so neglect
    if dark >= light then
        reading = reading + (dark - light)
    endif
next i  

'' Compute mean
reading = reading / N

goto mainloop
\end{lstlisting}






\subsection{Testing}
\begin{figure}[htb]
	\centering
	\includegraphics[width=0.8\linewidth]{old probe comparison.png}
	\caption{Comparison of old and new probe readings when aimed at a sheet of white printer paper, at a separation of 1, 2, 3, 4 and 5 cm. The red line indicates the boundary between high- and low-fat livers measured by the old probe in clinical trials. Plotted on a log-log scale.}
	\label{fig: old probe comparison}
\end{figure}

The measurement system was first tested to measure the correlation between readings taken with the old probe and those taken with the new design. This was important, because the new design's readings need to be highly correlated with the old probe to ensure that the new probe will have the same predictive power in clinical trials. The probes give an output in arbitrary units which is simply the ADC output of the PIC. The two probes had their correlation tested by measuring the output at fixed distances from a sheet of A4 paper. This will give different intensities of backscattered light, so can be used for calibration. Figure \ref{fig: old probe comparison} shows the outcome of this experiment. The R\textsuperscript{2} correlation coefficient is 0.9859, so the probes are highly correlated as desired. \\

\begin{figure}[htb]
	\centering
	\includegraphics[width=0.8\linewidth]{liquids.png}
	\caption{Measurements made by the probe on different liquids, with a liquid depth of \SI{2}{\centi\metre} and probe range of \SI{3}{\centi\metre}.}
	\label{fig: liquids}
\end{figure}

The measurement system was then tested to determine its response to lipid concentration. Readings were taken from shining the probe at water, semi-skimmed milk (2\% fat), and full-fat milk (4\% fat). The milk will have suspended lipid droplets which can be used to model probe's response to fat concentration. Figure \ref{fig: liquids} shows that the probe reading does increase for the liquids with higher fat concentration.






