\section{System Architecture}

The system was subdivided into two devices: the ``base'' unit and the ``remote'' unit. The remote unit would perform the measurement on the liver and transmit the data to the base unit. Only the remote unit will need to be hermetically sealed and able to withstand the sterilisation process, as the base unit does not encounter biological tissues. Furthermore, all the expensive electronics such as the screen can be contained within the base, making the remote unit cost less and hence be more disposable.\\

The remote unit will be packaged by casting the electronics in a resin, which will provide the hermetic seal and mechanical robustness. Because of this, it cannot be opened or have any sockets to connect cables. Therefore, it must communicate wirelessly with the base unit and be battery-powered with wireless charging. The wireless communications are required to operate over a range of several metres to cover the length of the operating theatre. Figure \ref{fig: architecture} shows the system block diagram with all the key subsystems labelled.

\begin{figure}[htb]
	\centering
	\includegraphics[width=\linewidth]{architecture.png}
	\caption{Block diagram of the system architecture. Key subsystems are: measurements (orange), power (green) and communications (yellow).}
	\label{fig: architecture}
	\vspace{-8mm}
\end{figure}