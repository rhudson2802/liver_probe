\section{Microcontroller}
A Peripheral Interface Controller (PIC) was chosen as the microcontroller for both the base and remote unit. PICs are versatile, with highly configurable pin assignments, which were suitable for sub-systems used in the product. Furthermore, they are easy to program in PICBASIC using the MicroCode Studio \cite{microcode_studio} and MPLAB IDE software packages. PICBASIC provides a high-level way to access the PIC's instruction set, with commands available which combine many complex assembly code instructions. The free version of MicroCode Studio limits the range of devices which can be programmed, so the PIC16F688 (16-pin) \cite{pic16f688} was chosen for the remote unit and the PIC18F2550 (28-pin) \cite{pic18f2550} for the base unit. These had the right number of I/O pins for their respective unit, as well as some special features which will be discussed in later sections. PICs are also relatively cheap, which is a good property for the disposable remote unit.\\

\begin{figure}[htbp]
	\centering
	\includegraphics[width=0.6\linewidth]{pickit3.PNG}
	\caption{PIC connected to PICkit 3 programmer with external circuitry \cite{pickit3}.}
	\label{fig: pickit3}
\end{figure}

The PICkit 3 \cite{pickit3} was used to program the PICs. Figure \ref{fig: pickit3} shows the relevant connections required for programming. A \SI{10}{\kilo\ohm} pull-up resistor was selected for the $\overline{\text{MCLR}}$ line. This is required because the PIC resets whenever $\overline{\text{MCLR}}$ is low, so it needs to be pulled high in normal operation. \\

The PIC18F2550 had a USB interface, which took up the majority of PORTC. When the PIC was tested on the breadboard, these USB pins could not operate as standard I/O pins as they pulled the pin high or low. The data sheet specified that USB functionality could be turned off by writing to the UCON and UCFG registers, but this did not work. This meant that the pins designated for USB were not used in this project.\\
