\section{Hardware Design}

A Peripheral Interface Controller (PIC) was chosen as the microcontroller for both the base and remote unit. PICs are versatile, with highly configurable pin assignments, which was suitable for the different types of sub-system which would be used in the product. Furthermore, they are easy to program in PICBASIC using the MicroCode Studio and MPLAB IDE software packages. PICBASIC provides a high level way to access the PIC's instruction set, with commands available which combine many complex assemble code instructions. The free version of MicroCode Studio limits the range of devices which can be programmed, so the PIC16F688 \cite{pic16f688} was chosen for the remote unit and the PIC18F2550 \cite{pic18f2550} for the base unit. These had the right number of I/O pins for their respective unit, as well as some special features which will be discussed in later sections. PICs are also relatively cheap, which is a good property for the disposable remote unit.\\

\subsection{Measurement System}
\begin{figure}[htbp]
	\centering
	\includegraphics[width=\linewidth]{measurement.PNG}
	\caption{Circuit schematic of diffuse reflectance measurement system}
	\label{fig: measurement schematic}
\end{figure}

Figure \ref{fig: measurement schematic} illustrates the circuit used to perform the backscatter measurements on the liver. Fundamentally, this circuit shines an LED at the liver, then produces an output voltage proportional to the amount of light reflected back. The clinical trials of the original probe \cite{Robertson} used two LEDs, one with a red wavelength (\SI{660}{\nano\metre}) and one with an infra-red wavelength (\SI{850}{\nano\metre}). Two photodiodes were also used, so over all 4 different measurements could be made (with each photodiode-LED combination). The results showed that the correlation with liver PNF did not depend on the wavelength or LED-photodiode spacing, so this was not a critical design requirement of the new probe. Therefore, an LED in the near infra-red (NIR) range was selected (Vishay TSHA4401 \cite{tsha4401} \SI{875}{\nano\metre}) along with a phototransistor also in the NIR range (Vishay BPW85B \cite{bpw85b} \SI{850}{\nano\metre}).\\

The left hand side circuit of figure \ref{fig: measurement schematic} provides a constant current to the LED. This is essential to ensure the measurements are repeatable, as the brightness of an LED is proportional to its current. The operation of the circuit will now be discussed. First consider the case where MOSFET Q1 has its gate driven low, so it is off. This means that negligible current will flow through the MOSFET, so the op-amp U1A has \SI{2.5}{\volt} at its non-inverting input. Assuming the op-amp is ideal (which means it has infinite gain and input impedance, and zero output impedance), the voltage at the inverting input will also be \SI{2.5}{\volt}. This means there is a constant \SI{2.5}{\volt} across R2, so it draws a constant \SI{11.4}{\milli\ampere}. Because the op-amp is assumed to be ideal, no current can go into the input pins, so the full \SI{11.4}{\milli\ampere} must be sourced from the op-amp output pin and hence pass through the LED D5. This creates a constant current source for the LED, assuming a constant \SI{2.5}{\volt} rail and that the ideal op-amp assumption is valid. When the MOSFET gate is driven high it turns on. This leads to the MOSFET having a very low impedance, so the non-inverting input is effectively tied to ground. Therefore R2 has no volts across it, and hence no current, so the LED turns off.\\

The MCP6002-I/SN \cite{mcp6002} was selected as the op-amp. This has a short circuit output current of \SI{23}{\milli\ampere}, so will comfortably be able to provide the desired \SI{11.4}{\milli\ampere}. Its input offset voltage is $\pm\SI{4.5}{\milli\volt}$, which leads to a potential current error of \SI{20}{\micro\ampere} (or 0.17\%), which is a negligible error. The input voltage noise density is \SI{28}{\nano\volt\per\sqrt{\hertz}} and the gain-bandwidth product is \SI{1}{\mega\hertz} so the worst case bandwidth is \SI{1}{\mega\hertz} (which assumes unity gain). This leads to an input noise voltage of \SI{28}{\micro\volt}, which again is negligible. The current source was tested for its invariance to temperature. When exposed to the heat fro ma hairdryer, the LED current only changed by 0.85\%, so there is confidence that the current source will be stable over the operating temperatures.\\

The light transmitted by the LED is then reflected off the target liver, and the received light is collected by phototransistor Q2. This produces a current proportional to the input light power. A transimpedance amplifier is used convert this to a readable voltage signal. Again assuming op-amp U1B is ideal implies that the voltage at the input pins are equal, and no current enters the pins. The phototransistor current $i_f$ therefore passes through the resistor R3, which leads to an output voltage signal $v_0 = 2.5 - i_fR_f$. The \SI{2.5}{\volt} rail at the non-inverting input acts to bias the op-amp, and was selected as half the voltage rails to ensure that the op-amp output signal could have maximum voltage swing before clipping at the supply rails.\\

\begin{table}[htbp]
	\centering
	\caption{Results for feedback resistor calibration experiments. The peak to peak signal is the difference between the output voltage when the LED was on and off, and the minimum signal is the output voltage when the LED was on.}
	\label{tab: tia feedback resistor}
	\begin{tabular}{|c|c|c|c|}
		\hline
		\textbf{LED Current} & \textbf{Feedback Resistor} & \textbf{Peak to Peak Signal} & \textbf{Minimum Signal}\\
		(mA)	&	(\si{\kilo\ohm})	&	(V)	&	(V)\\
		\hline
		\multirow{4}{*}{10}	&	2.2	&	0.93	&	1.20\\
						\cline{2-4}
						&	3.3	&	1.20	&	1.06\\
						\cline{2-4}
						&	4.7	&	1.60	&	0.62\\
						\cline{2-4}
						&	6.8	&	2.22	&	0.00\\
		\hline
		\multirow{4}{*}{20}	&	1.2	&	0.84	&	1.42\\
						\cline{2-4}
						&	1.5	&	1.07	&	1.15\\
						\cline{2-4}
						&	2.7	&	1.9	&	0.25\\
						\cline{2-4}
						&	3.3	&	2.09	&	0.00\\
		\hline
	\end{tabular}
\end{table}

The feedback resistor $R_f$ needed to be chosen to ensure good output characteristics. If it was too small, then the PIC's ADC would not be able to discriminate between changes in the light intensity.  The PIC16F688 uses a 10-bit ADC, so the maximum voltage resolution is \SI{4.88}{\milli\volt} when operated at \SI{5}{\volt}. The feedback resistor also cannot be too large, as this could lead to the op-amp saturating at the ground supply rail. Table \ref{tab: tia feedback resistor} shows the results of an experiment carried out to investigate the signal levels with different feedback resistors. The circuit was constructed on breadboard, with the LED and phototransistor soldered to stripboard to ensure they remained at a fixed distance of \SI{7.62}{\milli\metre}. This was then shone at a piece of A4 paper at a distance which led to the largest signal. The peak to peak signal is the difference between the transimpedance amplifier output voltage when the LED was on and off, and the minimum signal is the output voltage when the LED was on. Whenever the minimum signal was \SI{0}{\volt}, the amplifier had saturated so these values of $R_f$ are unsuitable. Therefore, the optimum $R_f = \SI{4.7}{\kilo\ohm}$, as this had the largest peak to peak signal without saturating. The system was tested using an LED current of \SI{20}{\milli\ampere} as well, but it was decided that this was too close to the op-amp output current limit to be a reliable, stable current source.\\

%LTSpice simulation

The transimpedance amplifier is a notoriously unstable circuit due to the phototransistor's junction capacitance. This can be stabilised by adding the capacitor C1 \cite{tia_stability}, which adds a zero into the feedback factor, compensating for the pole created by the phototransistor capacitance \cite{tia_stability}. The capacitor value was selected to be \SI{470}{\pico\farad}, as this gave a bandwidth of \SI{72}{\kilo\hertz} which is well above the required bandwidth of the circuit (which is \SI{10}{\kilo\hertz} at most), but is well below the gain-bandwidth product of \SI{1}{\mega\hertz} so it avoids the stability problems.\\

A button also needed to be designed for the probe, so the surgeon could indicate that they wanted to take a measurement. A standard push button could not be used because the remote unit had to be hermetically sealed. Therefore, a non-contact means of sensing a button press had to be devised. This could either be capacitive, sensing the finger like the touch screen on a mobile phone, or optical, by measuring light reflected back from the finger. It was found that the liver backscatter measurement system (figure \ref{fig: measurement schematic}) was discriminated well between the presence or absence of a finger at a range of a few centimetres, so this circuit was included twice within the remote unit to be used as a button. The downside to this method is that it is an active sensing system, requiring power to be supplied to the LED and phototransistor. This will limit the time the remote unit can operate over one full charge. \\

\begin{figure}[htbp]
	\centering
	\includegraphics[width=0.6\linewidth]{2-5v rail.PNG}
	\caption{Op-amp unity buffer used to generate \SI{2.5}{\volt} rail.}
	\label{fig: 2.5v rail}
\end{figure}

The measurement system used a \SI{2.5}{\volt} reference rail. This rail had to be stable as it provided the reference signal for the LED current source and transimpedance amplifier, so any variations in the \SI{2.5}{\volt} signal would lead to measurement errors. An op-amp unity buffer circuit (figure \ref{fig: 2.5v rail}) was used to generate this signal from a potential divider reference. This is more stable than directly using a potential divider, as any loading of the potential divider will decrease the reference voltage it provides. Because an op-amp draws negligible current, the resulting output voltage will be much more stable. The output will only remain stable the circuits it is connected to do not draw more than the op-amp's rated output current. The \SI{2.5}{\volt} rail is only used to supply current to a \SI{10}{\kilo\ohm} resistor (\SI{250}{\micro\ampere}) and op-amp input (negligible current), so the op-amp will not have to provide significant current. Additionally, the output will only be a fixed \SI{2.5}{\volt} reference if the \SI{5}{\volt} supply also remains stable, which is ensured by using a voltage regulator (see section \ref{power}).\\



\subsection{Communications}
The Quasar QAM-TX3 \cite{qam-tx} and QAM-RX10-433 \cite{qam-rx} transmitter/receiver modules were used to implement the wireless link. These had all the amplification, demodulation, filtering, and other signal processing required for a radio link fully integrated. Fully integrated modules were selected because they were cheap and simplified the design greatly, as the design only had to focus on low frequencies and not the complex effects associated with RF electronics. The Quasar modules allowed the raw data to be input to the data pin in baseband, then modulated this onto a \SI{433}{\mega\hertz} carrier using on-off keying (OOK). The transmitter then sent this modulated signal to its antenna, and was received by the receiver which demodulated the signal and output the baseband signal at its data pin. The modules could work up to \SI{3}{\kilo\bit\per\second}, which is a suitable speed for the small packets of data which will need to be sent by the remote unit. They operate in the \SI{433}{\mega\hertz} band, which is an unlicensed ISM band in the UK \cite{ism_band} so there should be no regulatory issues using this frequency. A \SI{100}{\pico\farad} decoupling capacitor was used between the power rails of both modules, to smooth out any ripple in the power supply to ensure a reliable RF stage.\\
%AM vs FM

\subsection{Power and Wireless Charging} \label{power}
The remote unit was designed to run from a \SI{5}{\volt} supply. \SI{3.3}{\volt} was not suitable as the LEDs had a forward voltage of \SI{1.5}{\volt}, so a \SI{3.3}{\volt} supply would potentially be too small for the current source designed in figure \ref{fig: measurement schematic}. Therefore, a set of batteries had to be selected which would provide this voltage. The batteries had to be small to ensure the overall remote unit design was small enough to fit in the surgeon's hand, and needed to be rechargeable because of the hermetic seal. Two different chemistries were available from RS Components \cite{rs} in a small, rechargeable package: lithium ion (Li-ion) and Nickel Metal Hydride (NiMH). The NiMH chemistry was more suitable due to the increased safety. Li-ion batteries are highly sensitive to over/under-voltage and high temperatures, and can blow up if these limits are exceeded \cite{batteries}. This requires complex protection circuitry and may mean that the battery will not survive being placed in an autoclave. NiMH are much more resilient to over and under charging, so are much safer than Li-ion. They also have a higher energy density \cite{batteries}. Nevertheless, they are lower voltage so more cells will be required to provide the desired \SI{5}{\volt}, have a long charging time, and they have a high self-discharge rate \cite{batteries}. This is not such an issue for this application, as the remote unit can remain charging until it is required for an operation, and then it will only be used for a short time before being placed on charge again. A \SI{2.4}{\volt} \SI{80}{\milli\ampere\hour} RS-Pro NiMH button cell \cite{rs_pro_batteries} was selected as suitable. Three of these were used to give a nominal \SI{7.2}{\volt} supply, which gave a margin of 44\% for when the cells started to discharge.\\

A power converter then needed to be selected to drop the \SI{7.2}{\volt} battery voltage to the required \SI{5}{\volt} rail. A Low-Dropout (LDO) voltage regulator was deemed suitable for this application, as it would provide a stable voltage rail. The advantage over a buck converter is the smaller device size and design simplicity, but this comes at the cost of worse efficiency. Furthermore, the output voltage will not have the high frequency ripple associated with switch-mode converters, which is desirable to create a stable analogue rail to ensure that there is no measurement errors. The MCP1702T-5002E/CB \cite{mcp1702} in a SOT-23A package was suitable. This was a \SI{5}{\volt} regulator with a rated output current of \SI{250}{\milli\ampere}. This was well above the \SI{50}{\milli\ampere} maximum current required by the remote unit (see section \ref{power budget}). The quiescent current was \SI{2}{\micro\ampere}, which is negligible compared with the remainder of the circuit, so the regulator will not affect the battery lifetime. The regulator will dissipate a maximum of $(\SI{7.2}{\volt}-\SI{5}{\volt})\times\SI{50}{\milli\ampere}=\SI{110}{\milli\watt}$. The SOT-23A package has a junction to air thermal resistance of \SI{336}{\celsius\per\watt}, so the maximum temperature increase is \SI{37}{\celsius}. In fact, the device will rarely operate at \SI{50}{\milli\ampere}, so the regulator will remain at ambient temperature.\\

\begin{figure}[htbp]
	\centering
	\includegraphics[width=\linewidth]{h bridge charger.PNG}
	\caption{Circuit schematic for H-Bridge inductive wireless charger. The left hand side circuit is included in the base unit, and the right hand side circuit is included in the remote unit.}
	\label{fig: charger schematic}
\end{figure}
	

Inductive charging was selected as an appropriate method of recharging the batteries. This involved placing two air-cored coils in close proximity, applying an AC voltage to the primary and rectifying the voltage across the secondary. Figure \ref{fig: charger schematic} shows the circuit schematic, which uses two PMOS and NMOS pairs in a H-bridge arrangement to generate the AC voltage. The MOSFETs are arranged in a CMOS inverter structure, so when the input gate voltage is high, their output is low, and when the gate is low their output is high. The two half bridges are driven with inverted signals, so the voltage across the load alternates between \SI{\pm 5}{\volt}. The series capacitor is used to create a resonant circuit with the inductor, which increases the voltage across the primary coil and smooths the square wave input voltage waveform to a sinusoid across the inductor. The capacitor is designed to give the circuit a resonant frequency equal to the drive frequency, to get the maximum gain possible. The series resistor is included to limit the current through the circuit. An ideal series LC circuit has zero impedance at resonance, so it will look like a short circuit to the external circuit. The PMOS transistors had a current rating of \SI{230}{\milli\ampere}, so using a series resistance of \SI{22}{\ohm} will limit the current to \SI{227}{\milli\ampere}, with the channel resistance of the MOSFETs and parasitic LC resistances acting to reduce it further.\\

The primary coil will generate an alternating magnetic field due to the sinusoidal voltage across it. This couples with the secondary coil to induce an alternating voltage. The coupling of air-cored transformers is very poor, so a large voltage will be required across the primary coil to ensure the secondary coil receives enough volts. The resonant circuit aids this. The secondary coil's alternating voltage is then rectified to a DC voltage by a diode bridge. The NSR05F20NXT5G Schottky barrier diode \cite{original_diode} was used in the diode bridge. These were chosen because of their very low forward voltage, \SI{0.2}{\volt}, which would make sure that the required secondary voltage was kept as small as possible. A larger diode drop would mean a larger secondary voltage was required, which could cause problems under the low coupling conditions of the air-cored transformer. Their reverse leakage was \SI{2}{\micro\ampere}, which was also good. They were rated to \SI{20}{\volt} and \SI{500}{\milli\ampere}, which were well below the expected operating conditions of \SI{10}{\volt} and \SI{10}{\milli\ampere}. The battery voltage was \SI{7.2}{\volt}, so the total required voltage across the secondary including the two diode drops was \SI{7.6}{\volt}.\\

\begin{figure}[htbp]
	\centering
	\includegraphics[width=\linewidth]{coil inductance.png}
	\caption{Measured coil inductance for different numbers of coil turns}
	\label{fig: coil inductance}
\end{figure}

The inductance of a solenoid is given by 
\begin{equation}\label{eq: solenoid}
L = \frac{\mu_0 N^2 A}{l}
\end{equation}
where $\mu_0$ is the permeability of free space, $N$ is the number of turns, $A$ is the cross-sectional area of the solenoid and $l$ is the length of the solenoid. The circular coils constructed for the wireless charger had an inner diameter of \SI{5}{\centi\metre}, and an average length of \SI{0.5}{\centi\metre} (this varied depending on the number of turns). Substituting these geometry parameters in equation \ref{eq: solenoid} implies that $L=\num{4.93e-7}N^2$. The coil inductance was then experimentally determined by measuring the resonant frequency of an LC circuit with known capacitance. The signal was a sinusoid generated from a PicoScope 2204A \cite{picoscope} with a series resistor at the output. The frequency was varied until the maximum output voltage across a parallel LC tank was measured. After this, the capacitance required for the circuit to be resonant at \SI{100}{\kilo\hertz} was calculated, by $C = 1/L(2\pi f_\text{res})^2$. Figure \ref{fig: coil inductance} shows the variation of coil inductance with the number of turns. The inductance follows a quadratic dependence on the number of turns, as is to be expected from equation \ref{eq: solenoid}. However, the coefficient is 10x smaller than equation \ref{eq: solenoid} predicts. This could be because equation \ref{eq: solenoid} assumes the flux within the solenoid is uniform, which may not be the case for the circular planar coil as the short length may lead to lots of fringing effects, which could influence the flux distribution. Furthermore, the construction of the coils meant that there was not a uniform distribution of the coil turns and there was significant overlap of the turns, which could also influence the measured inductance.\\

\begin{figure}[htbp]
	\centering
	\includegraphics[width=0.6\linewidth]{coil experiment.png}
	\caption{Setup for coil experiments}
	\label{fig: coil experiment setup}
\end{figure}
\begin{figure}[htbp]
	\centering
	\includegraphics[width=\linewidth]{circular coupling.PNG}
	\caption{The experimentally derived coupling coefficient of circular coils as a function of coil separation. The legend gives the turns ratio for the different experiments.}
	\label{fig: circular coupling}
%*******************REPLOT********************
\end{figure}

The performance of the proposed charging circuit was investigated. First, the coupling coil was determined as a function of the separation between the coils. The experiment was performed using two planar circular coils, where the primary coil had 100 turns and the secondary coil had 160 turns. The coils were centred on the same out-of-plane axis. Figure \ref{fig: coil experiment setup} illustrates the experiment setup. The primary coil was driven by the H-bridge circuit in figure \ref{fig: charger schematic}, and the secondary coil had its open circuit voltage measured. If the transformer were ideal, the secondary coil's voltage should be $V_2 = V_1 \times N_2 / N_1$ where $V_1$ is the voltage across the primary coil, and $N_2 / N_1$ is the secondary to primary turns ratio. To account for the poor coupling in an air-cored transformer, a coupling coefficient $k$ is introduced, so $V_2 = k \times V_1 \times N_2 / N_1$. The reduction in coupling is because the air core has a low permeability, so the flux from the primary is not strongly linked to the secondary, leading to a large leakage flux. The experiment was repeated for different turns ratios, all giving similar values of the coupling coefficient.. The experimentally determined coupling coefficients are plotted in figure \ref{fig: circular coupling}, which shows a coupling coefficient of 0.28 at a separation of \SI{10}{\milli\metre}.\\

%Coil inductance calculations

\begin{figure}[htbp]
	\centering
	\includegraphics[width=\linewidth]{charger simulation.PNG}
	\caption{LTspice simulation of charger circuit}
	\label{fig: charger simulation}
\end{figure}

The circuit was simulated in LTspice \cite{ltspice}, with the circuit schematic shown in figure \ref{fig: charger simulation}. A voltage source with series resistance was used to model the H-bridge output. The circuit was tested on breadboard, so the surface mount NSR05F20NXT5G diodes could not be used. Instead, a through-hole diode was selected for testing purposes (Vishay UF4001-E3/54 \cite{tht_diode}), and this diode was modelled in the LTspice simulation, so the results could be compared with the experimental results. The coupling between the coils was set to 0.28, to model \SI{1}{cm} coil separation. The inductors and batteries had an equivalent series resistance modelled as well. The simulation suggested that the primary coil should have an RMS voltage of \SI{44.4}{\volt}, and the secondary coil should have an RMS voltage of \SI{8.24}{\volt}. This led to an average battery charging current of \SI{13.1}{\milli\ampere}, which would be suitable to recharge the remote unit. %************EXPERIMENT RESULTS**************\\

%Coil combinations
%12V charger
%MOSFET drivers
% New diodes
 

\begin{figure}[htbp]
	\centering
	\includegraphics[width=\linewidth]{battery current sensor.PNG}
	\caption{Battery current sensor.}
	\label{fig: battery current sensor}
\end{figure}

A current sensor was included to measure the remote unit's charging and discharging currents. A \SI{1}{\ohm} resistor (R2 in figure \ref{fig: charger schematic}) was placed in series with the batteries to develop a voltage proportional to the current through the batteries. R2 was placed between BT2 and BT3 so it sat at \SI{2.4}{\volt}, which was important to ensure that the voltage input to the amplifier was not close to the amplifier supply rails, which would have resulted in clipping otherwise. The voltage across R2 was used as the input to a differential amplifier (figure \ref{fig: battery current sensor}). The output voltage $v_o = 2.5 - 47v_\text{sense} = 2.5 - 47 i_\text{bat}$, where $v_\text{sense}$ is the voltage across the current sense resistor R2, which in turn is equal to the battery current $i_\text{bat}$ because a \SI{1}{\ohm} resistor is used. The gain (47) is given by the ratio of resistors R20 and R17 (or equivalently R19 and R18) in figure \ref{fig: battery current sensor}. The \SI{2.5}{\volt} rail was used as a pseudo-ground for the non-inverting input to bias the output voltage at \SI{2.5}{\volt}, which would then be able to swing in both the positive and negative direction without clipping. If the real ground rail were used, the output voltage would be \SI{0}{\volt} for all discharging currents as the output cannot swing negative from a unipolar supply. The output signal was input to a PIC ADC pin to allow its value to be determined by the firmware.\\
%Differential amplifier gain calculation?



A circuit was also designed to monitor the voltage of the batteries, importantly to detect an under-voltage condition. As the batteries discharged, their voltage also reduced. If the battery voltage fell below \SI{5.15}{\volt} (the voltage regulator's rated voltage plus the dropout voltage), then the regulator would be unable to provide \SI{5}{\volt} to the remaining circuit. If the \SI{5}{\volt} rail was reduced, then the LED current and phototransistor output signal would be reduced, which means that the measurements would be inaccurate. Therefore, the PIC needed a way to detect an under-voltage condition, so that it could stop taking measurements (which would be wrong data), and cut-off any high power systems to prevent further battery discharge.\\

\begin{figure}[htbp]
	\centering
	\includegraphics[width=\linewidth]{battery voltage sensor.PNG}
	\caption{Battery under-voltage sensor}
	\label{fig: battery voltage sensor}
\end{figure}

Figure \ref{fig: battery voltage sensor} shows the circuit used to detect the under-voltage condition. U5 is the MCP1702 LDO voltage regulator described at the start of section \ref{power}. U6B is an op-amp which acts as a comparator with hysteresis. The hysteresis is required to prevent oscillations. When the batteries have a current, their output voltage is reduced due to their internal impedance, so if the circuit immediately cut off the current when an under-voltage condition were detected, then the battery voltage would immediately rise due to the reduced current, so the comparator would then indicate that the voltage was fine, allowing the current to flow which would cause an oscillatory loop. With the hysteresis, the batteries must be charged to above the upper threshold voltage $v_{TH}$ before the comparator will indicate that the battery voltage has risen, so this threshold can be designed to give a stable voltage above \SI{5}{\volt} when the batteries are loaded. The upper threshold voltage was designed to be \SI{7.2}{\volt} (when the batteries were open circuit) and the lower threshold $v_{TL}$ was set to \SI{4.96}{\volt} (when the batteries were closed circuit). The battery voltage is halved by the potential divider R21 and R22, because it would otherwise be too large for a \SI{5}{\volt} comparator. The resistors R23, R24 and R26 were chosen to set the threshold voltages according to the design equations \cite{hysteresis}:

\begin{multicols}{2}
\begin{equation}
\frac{v_{TL}}{v_{TH} - v_{TL}} = \frac{R26}{R23}
\end{equation}

\begin{equation}
\frac{v_{TL}}{5 - v_{TH}} = \frac{R24}{R23}
\end{equation}
\end{multicols}

R25 was set to \SI{0}{\ohm} so an extra footprint would be placed on the PCB design to allow for fine tuning of the hysteresis, if the original design were found to not be suitable during testing. Diode D7 was included because the PIC pin the circuit was connected to, RA3, was also the $\overline{\text{MCLR}}$ pin used by the PICKit3 programmer. This pin raised to a high voltage during PIC programming. Diode D7 clamps the op-amp output to \SI{5}{\volt} to prevent damage during programming. The \SI{33}{\kilo\ohm} resistor allows the programmer to adjust the pin voltage independent of the op-amp, so the op-amp does not tie it high or low.\\


\subsection{Remote Unit}
%Temperature sensor
%PicKit programmer
%Power budget

\subsection{Base Unit}
%Memory 
%RTC
%LCD
%switches
%power budget