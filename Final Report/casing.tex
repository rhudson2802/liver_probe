\section{Casing Design}

\subsection{Remote Unit}
The remote unit will have its PCB enclosed in a resin cast, which will provide a hermetic seal. The resin must be a biocompatible material to ensure the device is safe to use in vivo and will be approved by the regulator. There are many commercially available biocompatible resins, as they are used in fields like dentistry, which commonly uses PMMA for casts \cite{biocompatible_resin}. The resin cast needs to allow light to pass to and from the LEDs and phototransistors. If a transparent resin is used, this will not be a problem. If an opaque resin is used, then two transparent plastic screens should be included at the ends to allow the transmission of light.\\

As the device will be sterilised in an autoclave, it will experience large changes in temperature. This means that the resin selected must have its coefficient of thermal expansion reasonably matched to the FR4 used in the PCB and to the electronic component casings. If they are not well-matched, then, when the device is heated, the different parts will expand at different rates. This will lead to stresses building up inside the device which could cause damage. When the final product is manufactured, it should be tested over many autoclave cycles to ensure that no damage occurs.\\

\begin{figure}[htb]
	\centering
	\begin{subfigure}[b]{0.4\linewidth}
		\includegraphics[width=\linewidth]{cuboid.png}
		\caption{Cuboid}
		\label{fig: cuboid}
	\end{subfigure}
	\begin{subfigure}[b]{0.4\linewidth}
		\includegraphics[width=\linewidth]{cylinder.png}
		\caption{Cylinder}
		\label{fig: cylinder}
	\end{subfigure}
	\begin{subfigure}[b]{0.4\linewidth}
		\includegraphics[width=\linewidth]{half cylinder.png}
		\caption{Half-cylinder}
		\label{fig: half-cylinder}
	\end{subfigure}
	\begin{subfigure}[b]{0.4\linewidth}
		\includegraphics[width=\linewidth]{final design.png}
		\caption{Final Design}
		\label{fig: final design}
	\end{subfigure}
	\caption{3D renders of the potential casing designs produced using SketchUp \cite{sketchup}}
	\label{fig: casings}
\end{figure}

The design of the casing shape underwent several iterations to improve the user experience. The iterations were rendered in CAD and modelled using Play-Doh to see how they felt to hold. The first design was a simple cuboid shape (figure \ref{fig: cuboid}), with dimensions large enough to enclose the PCB (which measured \SI{80x40x35}{\milli\metre}). This felt cumbersome to hold, and the corners felt uncomfortable. The design was therefore changed to a cylinder (figure \ref{fig: cylinder}) to make the edges smooth. This felt more comfortable, but the cylindrical shape could lead to issues with the wireless charger. Maximum power is transferred when the charging coils are parallel, and zero power is transferred when they are perpendicular. Due to the rotational symmetry of the cylinder, it will not be clear which orientation the surgeon should place the remote unit on the charging pad, and it could possibly roll over and misalign. Thus, a new half-cylinder design (figure \ref{fig: half-cylinder}) was made. The flat surface on the bottom makes it clear which side should be placed against the charging pad. Furthermore, if the half-cylinder is placed on its curved side, gravity will rotate the device so the flat surface is pointing upwards, so the coils will still be parallel, albeit at a greater separation. One problem noted with this design (and all the previous ones) was that it was not clear which side of the device should be placed against the liver and which side has the ``take-measurement'' button. Therefore, the final design (figure \ref{fig: final design}) included an extrusion on the front of the device which should be held against the liver. The device should be held like a syringe, and the surgeon can use their thumb to press the button on the back (see figure \ref{fig: real casing}). It was also noted that in order to house the PCB, the half-cylinder design needed to have a diameter of \SI{70}{\milli\metre}, which was too big. The final design used a more circular shape to reduce the overall radius, but still included a flat edge on the bottom.

\begin{figure}[htb]
	\centering
	\includegraphics[width=0.35\linewidth]{real casing.PNG}
	\caption{Model of the final casing design.}
	\label{fig: real casing}
\end{figure}





\subsection{Base Unit}\label{base casing}

\begin{figure}[htb]
	\centering
	\begin{subfigure}[t]{0.4\linewidth}
		\includegraphics[width=\linewidth]{base.jpg}
		\caption{Base unit enclosure}
		\label{fig: base enclosure}
	\end{subfigure}
	\begin{subfigure}[t]{0.4\linewidth}
		\includegraphics[width=\linewidth]{charging pad.png}
		\caption{External charging pad which will be connected to the main base unit by an umbilical cable.}
		\label{fig: charging pad}
	\end{subfigure}
	\caption{}
\end{figure}

The base unit was packaged in a Hammond 1598ABK enclosure \cite{hammond}. This measured \SI{201x135x51}{\milli\metre}, which was just large enough to house the base PCB, LCD, buttons, and connectors. Holes were cut in the enclosure so these features could be accessed (see figure \ref{fig: base enclosure}). Buying a pre-made enclosure was more convenient than building a custom one and gave a better-quality finish. This was a cost-effective method for small batch sizes, but once commercial production begins it may be worthwhile to transition to a custom-made enclosure. Due to the limited size ranges available for pre-made enclosures, there was not enough room to include the wireless charging pad within the base unit (as this led to an unnecessarily large box to give the surface area required). Therefore, the wireless charging pad will be a separate unit connected to the main base unit by an ``umbilical'' cable. The charging pad enclosure only needs to house the charging coil and was designed with a hole cut in it to place the remote unit (see figure \ref{fig: charging pad}). The hole will ensure that the charging coils are properly aligned to give maximum power transfer.
