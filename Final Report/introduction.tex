\section{Background}

The liver is a vital organ within the body, responsible for metabolic processes, blood detoxification, and parts of the immune system \cite{liver_background}. The NHS performed 922 liver transplants in the year 2019-20 \cite{nhs_transplants}. This project concerns the design of a probe to assess the quality of donor organs and evaluate if they are suitable to be used as a transplant. This probe is intended to replace older product which gave good predictions for liver failure, but had several issues. The new design needed to be:

\begin{itemize}[noitemsep, nolistsep]
\item Easy to use
\item Able to survive being dropped
\item Low-cost (meaning it is disposable if damaged)
\end{itemize}


\subsection{Liver Transplants}
One threat to the survival of liver transplant patients is hepatic steatosis. Macrovesicular steatosis occurs when a hepatocyte’s (liver cell’s) vacuole, which is made up of fat, takes up the majority of the cell cytoplasm, squeezing the nucleus to the periphery \cite{Imber2002} (see figure \ref{fig: livers}). This is a benign disease in the donor patient \cite{Imber2002}.\\

\begin{figure}[htbp]
	\centering
	\begin{subfigure}[b]{0.4\linewidth}
		\includegraphics[width=\linewidth]{lean liver.PNG}
		\caption{Lean liver \cite{Bruno2008}}
		\label{}
	\end{subfigure}
	\begin{subfigure}[b]{0.4\linewidth}
		\includegraphics[width=\linewidth]{fatty liver.PNG}
		\caption{Fatty liver \cite{Bruno2008}}
		\label{fig: fatty liver}
	\end{subfigure}
	\caption{Comparison of a lean liver to a fatty liver. The white voids in B are fat deposits, showing that the rest of the cell is constrained to a small volume.}
	\label{fig: livers}
\end{figure}	


Issues arise when steatotic livers are transplanted. Whilst the exact mechanisms are not fully understood, the cold preservation process is linked with damage to the liver tissue, making it likely to fail in the recipient \cite{Imber2002}. Moderate to high levels of steatosis are associated with unacceptably high rates of primary non-function (PNF) and early allograft dysfunction (EAD). Both PNF and EAD are key causes of mortality in liver transplant patients, with patients having a one in three chance of dying if PNF occurs \cite{Robertson}.\\

It is currently unacceptable to transplant severely steatotic livers due to their high failure rate. Unfortunately, a method to quantify the level of steatosis quickly and accurately does not currently exist. The most common method to assess the quality of a donor liver is the opinion of the surgeon who removes the organ. However, this is subjective, and the surgeon performing the transplant may not trust the assessment \cite{Robertson}. To gain a quantitative assessment, biopsies can be sent for histological analysis, but this is a slow process. The longer the liver is outside the body, the higher the risk of PNF and EAD, so this is not an ideal method. Previous studies \cite{McLaughlin2010} have attempted to assess the fat content of liver tissue using electrical and optical spectroscopy. The results from the optical spectroscopy led to the development of a probe \cite{Robertson} which gave a rapid assessment of the liver steatosis. \\


\subsection{Original Probe}

\begin{figure}[htbp]
	\centering
	\begin{subfigure}[b]{0.4\linewidth}
		\includegraphics[width=\linewidth]{original probe.PNG}
		\caption{Original probe \cite{Robertson}.}
		\label{fig: original probe}
	\end{subfigure}
	\begin{subfigure}[b]{0.4\linewidth}
		\includegraphics[width=\linewidth]{probe tip.PNG}
		\caption{Diagram of tip of original probe \cite{Robertson}.}
		\label{fig: probe tip}
	\end{subfigure}
	\caption{}
\end{figure}

This project focuses on the development of the original liver probe produced by Robertson et al. \cite{Robertson}. This probe used optical backscatter properties of the liver tissues to assess its fat content. Figure \ref{fig: probe tip} shows a schematic of the tip of the original probe, which was pressed against the liver tissue. Two LEDs (one red, one infrared (IR)) were shone down an optic fibre to illuminate the liver tissue. Some of the light was scattered back towards the probe and was collected by another optic fibre, leading to a photodiode for detection. Lipid droplets within the tissue are a major source of scattering, so the more fat in the liver the greater the amount of backscattered light \cite{McLaughlin}. This meant that backscatter measurements could provide an indication of the liver fat content.\\

The probe progressed to clinical trials, where it was found that the correlation between its reading and fat concentration was poor (R\textsuperscript{2} = 0.456), but the correlation with PNF/EAD was strong \cite{Robertson}. The strong predictive power for PNF/EAD meant the probe could give a rapid, quantitative, point-of-surgery test to indicate whether the liver will survive in the new patient. These promising results warranted the further development of the probe.\\

The probe had several issues, which this project aims to resolve. Due to the high-pressure environment of surgery, the surgeons often damaged the probe. Sources of damage included being dropped on the floor and being placed in preservation solution. This project will develop a new iteration of the original probe which is more robust to the stresses of the clinical environment.\\




